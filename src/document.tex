% !TEX spellcheck=en_US
\documentclass[a4paper]{article}
\usepackage[american]{babel}
\usepackage[T1]{fontenc}
\usepackage[utf8]{inputenc}
\usepackage{graphicx}
\usepackage[table]{xcolor}
\usepackage{caption}
\usepackage{subcaption}
\usepackage{csquotes}
\usepackage{hyperref}
\usepackage[acronym]{glossaries}
\usepackage{amsmath}
\usepackage{amssymb}
\usepackage{amsthm}
\usepackage{mathtools}
\usepackage{tcolorbox}
\usepackage{circuitikz}
\usepackage[backend=biber,style=phys]{biblatex}

\addbibresource{bibliography.bib}

% 255 = 1, ..., 0 = 0
\definecolor{lightgray}{rgb}{0.9725,0.9725,0.9725}

\DeclareMathOperator\rank{rank}
\DeclareMathOperator\Span{Span}
\DeclareMathOperator\Col{Col}
\DeclareMathOperator\Row{Row}
\DeclareMathOperator\Null{Null}
\DeclareMathOperator\adj{adj}
\DeclareMathOperator\trace{tr}
\DeclareMathOperator\diag{diag}
\DeclareMathOperator\Hom{Hom}
\DeclareMathOperator\Floor{floor}
\DeclareMathOperator\Ceil{ceil}

\DeclarePairedDelimiter{\abs}{\lvert}{\rvert}
\DeclarePairedDelimiter{\norm}{\lVert}{\rVert}
\DeclarePairedDelimiter{\ceil}{\lceil}{\rceil}
\DeclarePairedDelimiter{\floor}{\lfloor}{\rfloor}

\newcommand*{\pref}[2]{(#1 \ref{#2})}
\newcommand*{\defines}{\coloneqq}
\newcommand*{\bolddot}{\boldsymbol{\cdot}}
\newcommand*{\setbuild}{\:\middle|\:}
\newcommand*{\vertbar}{\rule[-1ex]{0.5pt}{2.5ex}}
\newcommand*{\horzbar}{\rule[.5ex]{2.5ex}{0.5pt}}
\newcommand*{\proofright}{"$\Rightarrow$" }
\newcommand*{\proofleft}{"$\Leftarrow$" }
\newcommand*{\notimplies}{\mathrel{{\ooalign{\hidewidth$\not\phantom{=}$\hidewidth\cr$\implies$}}}}
\newcommand*{\tolim}[2]{\underset{#1\to#2}{\xrightarrow{\hspace*{1.1cm}}}}
\newcommand*{\seqinfty}[1]{\tolim{#1}{\infty}}
\newcommand*{\domain}[1]{\mathcal{D}(#1)}
\newcommand*{\codomain}[1]{\mathcal{C}(#1)}
\newcommand*{\image}[1]{\text{Im}(#1)}
\newcommand*{\kernel}[1]{\text{Ker}(#1)}
\newcommand*{\grad}[1]{\text{grad}\,#1}
\newcommand*{\divergence}[1]{\text{div}\,#1}
\newcommand*{\curl}[1]{\text{curl}\,#1}
\newcommand*\diff{\mathop{}\!\mathrm{d}}
\newcommand*{\evalat}[3]{\left.#1\vphantom{\int}\right\rvert_{#2}^{#3}}
\newcommand*{\hati}{\hat{\imath}}
\newcommand*{\hatj}{\hat{\jmath}}
\newcommand*{\hatk}{\hat{k}}
\newcommand*{\inlinematrix}[1]{\left(\begin{smallmatrix}#1\end{smallmatrix}\right)}

\renewcommand{\Re}{\operatorname{\mathfrak{Re}}}
\renewcommand{\Im}{\operatorname{\mathfrak{Im}}}

\makeatletter
\renewcommand*\env@matrix[1][*\c@MaxMatrixCols c]{%
  \hskip -\arraycolsep
  \let\@ifnextchar\new@ifnextchar
  \array{#1}}
\makeatother


\setacronymstyle{short-long}
\makenoidxglossaries

% general
\newacronym{pfd}{PFD}{Partial Fraction Decomposition}
\newacronym{dne}{DNE}{Does Not Exist}
% calculus
\newacronym{wlog}{WLOG}{Without Loss of Generality}
% differential equations
\newacronym{folde}{FOLDE}{First Order Linear Differential Equation}
\newacronym{solde}{SOLDE}{Second Order Linear Differential Equation}
\newacronym{ode}{ODE}{Ordinary Differential Equation}
\newacronym{pde}{PDE}{Partial Differential Equation}


\begin{document}

\begin{titlepage}
    \centering
    {\scshape\Huge Advanced Systems \par}
    \par\vspace{1cm}
    \includegraphics[width=0.45\textwidth]{images/logo.png}\par
    \vspace{3cm}
    {\huge\bfseries Lecture Notes \par}
    \vspace{1cm}
    {\scshape\large Mazawa Shinonome \par}
    \vspace{1cm}
    {\large\today\par}
    \vfill
    \begin{abstract}
        This document is a compilation of lecture notes for intermediate mathematics.
        It is intended to serve as a study guide for members of the Advanced System
        organization.
    \end{abstract}
\end{titlepage}

\newpage

\printnoidxglossary[type=\acronymtype]

\newpage

\tableofcontents

\newpage

% === Begin Section Includes ===

\section{Introduction}

\begin{flushleft}
	The Advanced System organization is an open-source software community who's main
	purpose it is to develop modern and user-friendly utility programs. In order to
	develop advanced applications, the study of mathematics is of particular interests:
	It lays the groundwork for many other fields of studies such as physics or computer
	science.
\end{flushleft}

\begin{flushleft}
	While this document has been drafted with the best intents and purposes, there
	is no warranty on the correctness of its content. You can find the source code
	on \url{www.github.com/Advanced-Systems/lecture-notes} to compile a new PDF
	file from scratch using the most recent version of this document or raise an
	issue to report mistakes under the issues tab of GitHub.
\end{flushleft}


\section{Logic}

% ==============================================================================
% ==============================================================================
% ==============================================================================

\subsection{Truth Tables}\label{subsec-truth-table}

% TODO: Add description and explanation here (T=1, F=0)

\begin{table}[hbt!]
    \centering
    \rowcolors{1}{}{lightgray}
    \begin{tabular}{*{6}{c}}
        $(A$ & $\land$ & $B)$ & $\rightarrow$ & $C$ \\
           T &         & T    &               & T   \\
           T &         & F    &               & T   \\
           F &         & T    &               & F   \\
           F &         & F    &               & F   \\
    \end{tabular}
    \caption{Conjunction}\label{table-conjunction}
\end{table}

\begin{table}[hbt!]
    \centering
    \rowcolors{1}{}{lightgray}
    \begin{tabular}{*{6}{c}}
        $(A$ & $\lor$ & $B)$ & $\rightarrow$ & $C$ \\
           T &        & T    &               & T   \\
           T &        & F    &               & T   \\
           F &        & T    &               & T   \\
           F &        & F    &               & F   \\
    \end{tabular}
    \caption{Disjunction}\label{table-disjunction}
\end{table}

\begin{table}[hbt!]
    \centering
    \rowcolors{1}{}{lightgray}
    \begin{tabular}{*{6}{c}}
        $(A$ & $\Rightarrow$ & $B)$ & $\rightarrow$ & $C$ \\
           T &               & T    &               & T   \\
           T &               & F    &               & F   \\
           F &               & T    &               & T   \\
           F &               & F    &               & T   \\
    \end{tabular}
    \caption{Subjunction}\label{table-subjunction}
\end{table}

\begin{table}[hbt!]
    \centering
    \rowcolors{1}{}{lightgray}
    \begin{tabular}{*{6}{c}}
        $(A$ & $\Leftrightarrow$ & $B)$ & $\rightarrow$ & $C$ \\
           T &                   & T    &               & T   \\
           T &                   & F    &               & F   \\
           F &                   & T    &               & F   \\
           F &                   & F    &               & T   \\
    \end{tabular}
    \caption{Bisubjunction}\label{table-bisubjunction}
\end{table}

\begin{table}[hbt!]
    \centering
    \rowcolors{1}{}{lightgray}
    \begin{tabular}{*{6}{c}}
        $(\neg$ & $B)$ & $\rightarrow$ & $C$ \\
                & T    &               & F   \\
                & F    &               & T   \\
    \end{tabular}
    \caption{Negation}\label{table-negation}
\end{table}

% ==============================================================================
% ==============================================================================
% ==============================================================================

\subsection{Circuit Representation of Logical Operations}\label{subsec-circuits}

Truth tables are very useful in determining the nature of logic gates.
In \pref{Table}{subsec-truth-table} find defined the axioms which build the basis for
this section. Take \texttt{XOR}, for instance: this logic gate evaluates to $1$
if and only if one of both poles receives $1$ as an input as opposed to \texttt{OR}
which also accepts two positive-valued states favorably.

\begin{table}[hbt!]
    \centering
    \rowcolors{1}{}{lightgray}
    \begin{tabular}{*{6}{c}}
        $A$ & $B$ & $A\texttt{ OR }B$ & $A\texttt{ AND }B$ & $A\texttt{ NAND }B$ & $(A\texttt{ OR }B)\texttt{ AND }(A\texttt{ NAND }B)$ \\
          1 & 1   & 1                 & 1                  & 0                   & 0                                                    \\
          1 & 0   & 1                 & 0                  & 1                   & 1                                                    \\
          0 & 1   & 1                 & 0                  & 1                   & 1                                                    \\
          0 & 0   & 0                 & 0                  & 1                   & 0                                                    \\
    \end{tabular}
    \caption{\texttt{XOR} Truth Table}\label{truth-table-xor}
\end{table}

Based on this truth table it is possible to implement a logic gate that replicates
a \texttt{XOR} gate in the following way:

\begin{figure}[hbt!]
    \centering
    \begin{circuitikz}
        % gates
        \node[or port,draw] at (0,2) (or) {};
        \node[nand port,draw] at (0,0) (nand) {};
        \node[and port,draw] at (2,1) (and) {};
        % inputs
        \node at ($(or.in 1) + (-1,0)$) (A) {$A$};
        \node at ($(nand.in 2) + (-1,0)$) (B) {$B$};
        % wires
        \draw (A) -- (or.in 1);
        \draw (A) -- (nand.in 1);
        \draw (B) -- (or.in 2);
        \draw (B) -- (nand.in 2);
        \draw (or.out) -- (and.in 1);
        \draw (nand.out) -- (and.in 2);
        \draw (and.out) -- ($(and.out) + (0.5,0)$) node[anchor=west] (Q) {$Q$};
    \end{circuitikz}
    \caption{Circuit of an \texttt{XOR} gate}\label{circuit-xor-gate}
\end{figure}

A half adder is a circuit that adds two binary numbers, each one bit in size.
The result $S$ is also represented by a one bit value, so in case of 
$1_2+1_2=(10)_2$ the second digit must be carried over in $C$ (hence the name 
\textit{carrier}) in order to be preserved for future operations.

\begin{figure}[hbt!]
    \centering
    \begin{circuitikz}
        % xor gate
        \node[xor port,draw] at (0,2) (xor) {};
        \draw (xor.in 1) -- ($(xor.in 1) + (-0.75,0)$) node[anchor=east] (A) {$A$};
        \draw (xor.in 2) -- ($(xor.in 2) + (-0.75,0)$) node[anchor=east] (B) {$B$};
        \draw (xor.out) -- ($(xor.out) + (0.75,0)$) node[anchor=west] (S) {$S$};
        % and gate   
        \node[and port,draw] at (0,0) (and) {};
        \draw (xor.in 1) -- (and.in 1);
        \draw ($(xor.in 2) + (-0.25,0)$) -- ($(and.in 2) + (-0.25,0)$) -- (and.in 2);
        \draw (and.out) -- ($(and.out) + (0.75,0)$) node[anchor=west] (C) {$C$};        
    \end{circuitikz}
    \caption{Circuit of an half adder}\label{circuit-half-adder}
\end{figure}

\begin{table}[hbt!]
    \centering
    \rowcolors{1}{}{lightgray}
    \begin{tabular}{*{4}{c}}
        $A$ & $B$ & $S$ & $C$ \\
          1 & 1   & 0  & 1    \\
          1 & 0   & 1  & 0    \\
          0 & 1   & 1  & 0    \\
          0 & 0   & 0  & 0    \\
    \end{tabular}
    \caption{Half Adder Truth Table}\label{truth-table-half-adder}
\end{table}

As opposed to an half adder, a full adder takes one more input (a so-called \texttt{carry in})
and two outputs, \textit{i.e.} \texttt{carry out} and \texttt{sum}. To build a
more sophisticated adder, chain half adders in a way that allows the result of
the previous sum to be carried over as \texttt{cin} to the next half adder.

\begin{figure}[hbt!]
    \centering
    \begin{circuitikz}
        % gates
        \node[xor port,draw,anchor=center] at (0,4) (xor1) {D};
        \node[xor port,draw] at (2,4) (xor2) {};
        \node[and port,draw] at (2,2) (and1) {E};
        \node[and port,draw] at (2,0) (and2) {F};
        \node[or port,draw] at (4,1) (or) {};
        % inputs
        \node at ($(xor1.in 1) + (-1,0)$) (A) {$A$};
        \node at ($(xor1.in 2) + (-1,0)$) (B) {$B$};
        \node at ($(A) + (0,-1.5)$) (C) {$C_{in}$};
        % wires
        \draw (A) -- (xor1.in 1);
        \draw (B) -- (xor1.in 2);
        \draw (xor1.out) -- (xor2.in 1);
        \draw (C) -- (xor2.in 2);
        \draw (C) -- (and1.in 1);
        \draw (xor1.out) -- ($(and1.in 2)-(0.5,0)$) -- (and1.in 2);
        \draw ($(A)+(0.75,0)$) -- ($(A)+(0.75,-4.56)$) -- (and2.in 2);
        \draw ($(B)+(1,0)$) -- ($(B)+(1,-3.425)$) -- (and2.in 1);
        \draw (and1.out) -- (or.in 1);
        \draw (and2.out) -- (or.in 2);
        \draw (xor2.out) -- ($(xor2.out) + (0.5,0)$) node[anchor=west] (S) {$S$};
        \draw (or.out) -- ($(or.out) + (0.5,0)$) node[anchor=west] (Q) {$C_{out}$};
    \end{circuitikz}
    \caption{Circuit of an full adder}\label{circuit-full-adder:1}
\end{figure}

Circuit (\ref{circuit-full-adder:1}) introduced three additional truth values 
to make the truth table (\ref{truth-table-full-adder}) for the full adder easier
to read\footnote{Note that $S:\Leftrightarrow D\texttt{ XOR } C_{in}$ and 
$C_{out} :\Leftrightarrow E\texttt{ OR }F$}. It is helpful to think of gate $D$
and gate $F$ as the first half adder.

\begin{table}[hbt!]
    \centering
    \rowcolors{1}{}{lightgray}
    \begin{tabular}{*{8}{c}}
        $A$ & $B$ & $C_{in}$ & $A\texttt{ XOR }B$ & $D\texttt{ XOR }C_{in}$ & $D\texttt{ AND }C_{in}$ & $F$ & $E\texttt{ OR }F$ \\
          1 & 1   & 1        & 0                  & 1                       & 0                       & 1   & 1                 \\
          1 & 1   & 0        & 0                  & 0                       & 0                       & 1   & 1                 \\
          1 & 0   & 1        & 1                  & 0                       & 1                       & 0   & 1                 \\
          0 & 1   & 1        & 1                  & 0                       & 1                       & 0   & 1                 \\
          1 & 0   & 0        & 1                  & 1                       & 0                       & 0   & 0                 \\
          0 & 1   & 0        & 1                  & 1                       & 0                       & 0   & 0                 \\
          0 & 0   & 1        & 0                  & 1                       & 0                       & 0   & 0                 \\
          0 & 0   & 0        & 0                  & 0                       & 0                       & 0   & 0                 \\
    \end{tabular}
    \caption{Full Adder Truth Table}\label{truth-table-full-adder}
\end{table}


\section{Sets}\label{sec-sets}

% ==============================================================================
% ==============================================================================
% ==============================================================================

.....

\subsection{Indicator Functions}\label{subsec-indicator-functions}

\begin{definition}\label{def-indicator-function}
    Let $X$ be a set. For each subset $A \subseteq X$ there is defined the 
    indicator function\footnote{Sometimes, the term characteristic function is 
    used to describe the function that indicates membership in a set in other 
    fields of mathematics. The term indicator function is more prominently used 
    in probability theory.} $\chi_A:X\rightarrow \{0,1\}$ of the set $A$ as
    \begin{equation}
        \chi_A(x) = \begin{cases}
            1,\;\text{ if } x \in A \\
            0,\;\text{ if } x \notin A
        \end{cases}
    \end{equation}
    We will write $\chi_A$ instead of $\chi_A(x)$ if the parameter
    $x$ is unambiguous.
\end{definition}

\begin{thm}\label{thm-complement-indicator-function}
    Let $\chi_A$ be an indicator function. Then the complement of $A$ (i.e. 
    $\overline{A} \defines A^C$) is
    \begin{equation*}
        \chi_{\overline{A}} = 1 - \chi_A
    \end{equation*}
\end{thm}

\begin{proof}
    Of \pref{theorem}{thm-complement-indicator-function}.
    \begin{align*}
        \chi_{\overline{A}}(x) &= \begin{cases}
            1,\;\text{ if } \neg(x \in A) \\
            0,\;\text{ if } \neg(x \notin A)
        \end{cases} \\
        &= \begin{cases}
            1-0,\;\text{ if } x \notin A \\
            1-1,\;\text{ if } x \in A
        \end{cases} \\
        &=1 - \begin{cases}
            0,\;\text{ if } x \notin A \\
            1,\;\text{ if } x \in A
        \end{cases} \\
        &= 1 - \chi_A(x)
    \end{align*}
\end{proof}

\begin{thm}\label{thm-set-cardinality-indicator-function}
    Let $\chi_A$ be an indicator function with $A$ as a finite set. Then,
    \begin{equation*}
        \abs{A} = \sum_{x \in X} \chi_A(x)
    \end{equation*}
\end{thm}

\begin{proof}
    Of theorem (\ref{thm-set-cardinality-indicator-function}).
    \begin{align*}
        \abs{A} &=\sum_{x\in A}\underbrace{\chi_A(x)}_{1}\\
                &=\sum_{x\in A}\chi_A(x)+\sum_{x\in X\setminus A}\underbrace{\chi_A(x)}_{0}\\
                &=\sum_{x\in (A\cup (X\setminus A))}\chi_A(x)\\
                &\overset{(\star)}{=}\sum_{x\in X}\chi_A(x).
    \end{align*}
    since 
    \begin{equation*}
        (\star) :\Leftrightarrow A\cup(X\setminus A)=(A\cup X)\setminus(A\setminus A)=X\setminus \emptyset=X
    \end{equation*}
\end{proof}

\begin{thm}\label{thm-cap-indicator-function}
    Let $\chi_A,\chi_B$ be indicator functions. Then:
    \begin{equation*}
        \chi_{A \cap B} = \chi_A\chi_B
    \end{equation*}
\end{thm}

\begin{proof}
    Of \pref{theorem}{thm-cap-indicator-function}.
    \begin{align*}
        \chi_{A \cap B} &= \begin{cases}
            1\;\text{ if } x \in (A \cap B)\\
            0\;\text{ if } x \notin (A \cap B)
        \end{cases}\\
        &= \begin{cases}
            1\;\text{ if } x \in A \land x \in B\\
            0\;\text{ if } x \notin A \land x \notin B
        \end{cases}\\
        &= \left(\begin{cases}
            1\;\text{ if } x \in A\\
            0\;\text{ if } x \notin A
        \end{cases}\right)\left(\begin{cases}
            1\;\text{ if } x \in B\\
            0\;\text{ if } x \notin B
        \end{cases}\right)\\
        &= \chi_A\chi_B
    \end{align*}
\end{proof}

\begin{thm}\label{thm-cup-indicator-function}
    Let $\chi_A,\chi_B$ be indicator functions. Then:
    \begin{equation*}
        \chi_{A \cup B} = \chi_A + \chi_B - \chi_A\chi_B
    \end{equation*}
\end{thm}

\begin{proof}
    Of \pref{theorem}{thm-cup-indicator-function}.
    By using the fact that $\overline{\overline{A}}=A$ in combination with 
    \pref{theorem}{thm-complement-indicator-function} and
    \pref{theorem}{thm-cap-indicator-function} we get that
    \begin{align*}
        \chi_{\overline{\overline{A \cup B}}} 
        &= 1 - \chi_{\overline{A \cup B}}\\
        &\overset{(\star)}{=} 1 - \chi_{\overline{A} \cap \overline{B}}\\
        &= 1 - \chi_{\overline{A}} \chi_{\overline{B}}\\
        &= 1 - (1 - \chi_A)(1 - \chi_B)\\
        &= 1 - \left(1 - \chi_B - \chi_A + \chi_A\chi_B\right)\\
        &= \chi_A - \chi_B + \chi_A\chi_B\\
    \end{align*}
    Recall that $(\star)$ uses one of De Morgan's laws which states that
    \begin{align*}
        (\star) &\iff \overline{A \cup B}\\
                &\iff \bigwedge_{x\in X}\{x\notin (A\cup B)\}\\
                &\iff \bigwedge_{x\in X}\{x\notin A \land x\notin B\}\\
                &\iff \overline{A} \cap \overline{B}
    \end{align*}
\end{proof}

\section{Single Variable Calculus}

TODO: Add descriptions


% === End Section Includes ====

\newpage

\medskip
\printbibliography

\end{document}
