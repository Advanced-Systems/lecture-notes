\subsection{Method of Integrating Factors}\label{subsec-meth-of-int-factors}

\begin{flushleft}
	The method of integrating factors is an algorithm for solving differential equations
	of the form described in \pref{equation}{eq-folde-simple3}. If this differential
	equation were already in the form of \pref{equation}{def-separable-equation} it
	would be a separable differential equation whose exact solution could be immediately
	determined by integrating both sides. However, for $p(x)\neq0$ this is not an option
	until we find an \emph{integrating factor} $\mu(x)$ and multiply this function to
	\pref{equation}{eq-folde-simple3}. The reason why we would do that is that we did
	like to transform the left-hand of the equation below such that it can be written
	in terms of a derivative with respect to $x$ which is how the method of integrating
	factors implements a solution for a \gls{folde} written	in general form.
\end{flushleft}

\begin{equation*}
	\mu(x)\frac{\diff y}{\diff x} + \mu(x)p(x)y = \mu(x)g(x)
\end{equation*}

\begin{flushleft}
	From here on we can make the following observation: as far as the right-hand
	side of the	aforementioned equation is concerned, it would be beneficial to our
	objective to find a function $\mu(x)$ (the so called \emph{integrating factor})
	such that
\end{flushleft}

\begin{equation}\label{eq-meth-of-int-factors-tmp1}
	\mu(x)g(x) = \frac{\diff}{\diff x}\left(\mu(x)y\right)
\end{equation}

\begin{flushleft}
	Based on this assumption we further note that if this were to be the
	case, then we can simplify this expression with the product rule, so from
	\pref{equation}{eq-meth-of-int-factors-tmp1} it follows that
\end{flushleft}

\begin{align*}
	\mu(x)\frac{\diff y}{\diff x} + \mu(x)p(x)y & = \frac{\diff}{\diff x}\left(\mu(x)y\right) \\
	\implies \mu(x)p(x)                         & = \frac{\diff \mu(x)}{\diff x}
\end{align*}

\begin{flushleft}
	By separation of variables this differential equation reveals the integrating
	factor which takes on the form of
\end{flushleft}

\begin{align}
	\int \frac{\mu^\prime(x)}{\mu(x)} \diff x & = \int p(x) \diff x                 \nonumber                            \\
	\implies \ln\abs{\mu(x)} + C              & = \int p(x) \diff x                  \nonumber                           \\
	\overset{C=0}{\implies} \mu(x)            & = \exp\left(\int p(x) \diff x\right) \label{eq-meth-of-int-factors-tmp2}
\end{align}

\begin{flushleft}
	Notice that imposing $C=0$ on \pref{equation}{eq-meth-of-int-factors-tmp2}
	does not cause a loss of generality. Taking this into account, we solve
	\pref{equation}{eq-meth-of-int-factors-tmp1} for $\mu(x)$ which yields
	the final transformation
\end{flushleft}

\begin{equation}\label{eq-meth-of-int-factors}
	y = \frac{1}{\mu(x)}\int \mu(x)g(x)\diff x
\end{equation}

\begin{flushleft}
	The most noticeably constraint for this technique is that we limit ourselves
	to integrals that we are able to solve analytically which is something that is
	going to haunt is in the next sections.
\end{flushleft}
