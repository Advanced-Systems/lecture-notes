\subsection{Higher Order Derivatives}\label{subsec-higher-order-derivatives}

\begin{definition}\label{def-higher-order-derivatives}
	We define the $n$-th derivative by recursion as the $(n-1)$-th derivative of
	the function $f$ and denote it in Leibniz notation by
	\begin{equation}
		f^{(n)}(x) = \frac{\diff^n f}{\diff x^n}
	\end{equation}
	if $f$ is $n$-times differentiable.
\end{definition}

\begin{rem}
	The zeroth derivative of a function is just the original function.
\end{rem}

\begin{thm}\label{thm-higher-order-derivative-arithmetic}
	Let $f$ and $g$ be two differentiable functions, and $c\in\mathbb{R}$. Then
	\footnote{Recall that $\binom{n}{k}=\tfrac{n!}{k!(n-k)!}$}
	\begin{enumerate}
		\item $(c\cdot f)^{(n)}(x_0) = c \cdot f^{(n)}(x_0)$
		\item $(f \pm g)^{(n)}(x_0) = f^{(n)}(x_0) \pm g^{(n)}(x_0)$
		\item $(f \cdot g)^{(n)}(x_0) = \sum_{k=0}^n \binom{n}{k}f^{(k)}(x_0) \cdot g^{(k-1)}(x_0)$
	\end{enumerate}
	The third item in this list is also called Leibniz' formula.
\end{thm}

\begin{exm}
	Let $f(x)=x^2e^x$. Find the $9$-th derivative of this function.
	\begin{flushleft}
		\textbf{Answer}: By \pref{theorem}{thm-higher-order-derivative-arithmetic},
		we have that
		\begin{align*}
			(x^2e^x)^{(9)} & = \sum_{k=0}^9\binom{9}{k}(x^2)^{(k)}(e^x)^{(k)}                                                                                \\
			               & = \left(\binom{9}{0}x^2+\binom{9}{1}2x+\binom{9}{2}2\right)e^x &  & \text{\pref{example}{exm-derivative-of-inverse-function:1}} \\
			               & = \left(x^2+18x+72\right)e^x
		\end{align*}
	\end{flushleft}
\end{exm}

\subsubsection{Fermats Theorem for Extrema}\label{subsubsec-fermtas-theorem-extrema}

\begin{thm}\label{thm-fermats-theorem-extrema}
	Let $f(x)$ be a function defined on $(a,b)$ and let this function take on
	a maximum (or minimum) value at some point $c\in(a,b)$. Fermat's theorem for
	extrema states that if $f$ is differentiable, then
	\begin{equation}
		f^\prime(c)=0
	\end{equation}
\end{thm}

\begin{proof}
	Of \pref{theorem}{thm-fermats-theorem-extrema}.
	\begin{flushleft}
		\gls{wlog} assume that $x_0$ is a maximum. Since $f$ is differentiable
		we can write
		\begin{align*}
			f^\prime(x_0) & = \lim_{x \to x_0}\frac{f(x)-f(x_0)}{x-x_0}   \\
			              & = \lim_{x \to x_0^+}\frac{f(x)-f(x_0)}{x-x_0} \\
			              & = \lim_{x \to x_0^-}\frac{f(x)-f(x_0)}{x-x_0}
		\end{align*}
		Now consider all $x>x_0$; it follows that
		\begin{align}
			 & x-x_0>0\nonumber                                                                     \\
			\implies
			 & f(x)-f(x_0)\leq0\nonumber                                                            \\
			\implies
			 & \frac{f(x)-f(x_0)}{x-x_0}\leq0\nonumber                                              \\
			\implies
			 & \lim_{x \to x_0^+}\frac{f(x)-f(x_0)}{x-x_0}\leq0\label{eq-fermats-theorem-extrema:1}
		\end{align}
		Next consider where $x<x_0$; then
		\begin{align}
			 & x-x_0<0\nonumber                                                                     \\
			\implies
			 & f(x)-f(x_0)\leq0\nonumber                                                            \\
			\implies
			 & \frac{f(x)-f(x_0)}{x-x_0}\geq0\nonumber                                              \\
			\implies
			 & \lim_{x \to x_0^-}\frac{f(x)-f(x_0)}{x-x_0}\geq0\label{eq-fermats-theorem-extrema:2}
		\end{align}
		In summary, \pref{equation}{eq-fermats-theorem-extrema:1} and \pref{equation}{eq-fermats-theorem-extrema:2}
		imply that
		\begin{equation*}
			f^\prime(x_0)=0
		\end{equation*}
	\end{flushleft}
\end{proof}

\begin{rem}
	In high school we often used implicitly \pref{theorem}{thm-fermats-theorem-extrema}
	to find candidates for maxima and minima values. But this theorem doesn't find
	\textit{all} possible candidates. Take for instance the absolute value function.
	It obviously has a minimum in the origin, but Fermat's theorem for extrema is not
	going to pick up this candidate since $f(x)=\abs{x}$ is not differentiable at $x=0$.
	On the other side, this theorem may even yield candidates that neither maxima nor
	minima (take for instance $f(x)=x^3$ at $x=0$) so you have to use this theorem with
	caution. It is not by accident that these values are referred to as \enquote{candidates}.
	They may or may not promote to extrema depending on the actual function.
\end{rem}

\subsubsection{Rolle's Theorem}\label{subsubsec-rolles-theorem}

\begin{thm}\label{thm-rolles-theorem}
	Let $a,b\in\mathbb{R}$ where $a<b$. Let $f$ be a continuous in $[a,b]$ and
	differentiable in $(a,b)$ subject to the condition \cite[p.169]{wuest2009} that
	\begin{equation*}
		f(a)=f(b)
	\end{equation*}
	Then there exists a $c\in(a,b)$ such that
	\begin{equation}
		f^\prime(c)=0
	\end{equation}
\end{thm}

\begin{proof}
	Of theorem (\ref{thm-rolles-theorem}).
	\begin{flushleft}
		Since $f$ is continuous on $[a,b]$, by \pref{theorem}{thm-weierstrass-theorems}
		it has a minimum $m$ and a maximum $M$. If $m=M$, then $f$ must be a constant
		function in which case we are done (\textit{cf.} \pref{example}{exm-derivatives:1}).
		But if we rule out this possibility, then $m<M$ implies that at least one
		the these extrema is not obtained at the endpoints of this interval since the
		theorem assumes that $f(a)=f(b)$ but $f(m)\neq f(M)$. Denote the extrema
		that is not an endpoint by $c\in(a,b)$. Because the function is differentiable
		in $(a,b)$, by \pref{theorem}{thm-fermats-theorem-extrema} we know that $f^\prime(c)=0$.
	\end{flushleft}
\end{proof}

\subsubsection{Mean Value Theorem}\label{subsubsec-mean-value-theorem}

\begin{thm}\label{thm-mean-value-theorem}
	Let $a,b\in\mathbb{R}$ where $a<b$. Let $f$ be continuous in $[a,b]$ and
	differentiable in $(a,b)$. Then there exists a $c\in(a,b)$ such that
	\footnote{This is also known as Lagrange's theorem}
	\begin{equation}
		f^\prime(c)=\frac{f(b)-f(a)}{b-a}
	\end{equation}
\end{thm}

\begin{proof}
	Of \pref{theorem}{thm-mean-value-theorem}.
	\begin{flushleft}
		We define a new function $F$ such that
		\begin{equation*}
			F(x)=f(x)-\underbrace{\left(f(a)+\frac{f(b)-f(a)}{b-a}(x-a)\right)}_{\text{secant line}}
		\end{equation*}
		Then \pref{theorem}{thm-arithmetic-continuous}, $F$ is continuous on $[a,b]$
		because $f$ was also continuous. It's also differentiable on $(a,b)$ for
		precisely the same reason, \textit{cf.} \pref{theorem}{thm-derivative-arithmetic}.
		Notice also that $F(a)=F(b)=0$. So, by \pref{theorem}{thm-rolles-theorem} there
		exists a $c\in(a,b)$ such that $F^\prime(c)=0$. Then
		\begin{align*}
			F^\prime(c) & = f^\prime(c) - \frac{f(b)-f(a)}{b-a} \\
			            & = 0
		\end{align*}
		Hence,
		\begin{equation*}
			f^\prime(c)=\frac{f(b)-f(a)}{b-a}
		\end{equation*}
	\end{flushleft}
\end{proof}

\subsubsection{Cauchy's Theorem}\label{subsubsec-cauchys-theorem}

\begin{thm}\label{thm-cauchys-theorem}
	Let $f(x)$ and $g(x)$ be continuous in $[a,b]$ and differentiable in $(a,b)$.
	Assume that $\forall x\in(a,b)$, $g^\prime(x)=0$. Then Cauchy's theorem states
	that
	\begin{enumerate}
		\item $g(a)\neq g(b)$
		\item $\exists c\in(a,b)$ s.t.
		      \begin{equation}
			      \frac{f^\prime(x)}{g^\prime(x)}=\frac{f(b)-f(a)}{g(b)-g(a)}
		      \end{equation}
	\end{enumerate}
\end{thm}

\begin{thm}\label{thm-differentiable-constant-zero}
	Let $f$ be a differentiable function defined on $(a,b)$. The function is constant
	\textit{iff} for any $x\in(a,b)$, $f^\prime(x)=0$.
\end{thm}

\begin{proof}
	Of \pref{theorem}{thm-differentiable-constant-zero}.
	\begin{flushleft}
		\proofright: We already proved this direction in \pref{example}{exm-derivatives:1}.
	\end{flushleft}
	\begin{flushleft}
		\proofleft: \gls{wlog} take any two points $x,y\in(a,b)$ such that $a<x<y<b$.
		By \pref{theorem}{thm-mean-value-theorem} on $[x,y]$, there exists a point
		$c\in(x,y)$ such that
		\begin{equation*}
			f^\prime(c) = \frac{f(y)-f(x)}{y-x} = 0
		\end{equation*}
		Hence, $f(y)=f(x)$. Therefore, $f$ is a constant function.
	\end{flushleft}
\end{proof}

\begin{thm}\label{thm-differentiable-strictly-monotone}
	Let $f$ be a differentiable function on $(a,b)$. If $f^\prime(x)>0$ for any
	$x\in\domain{f}$, then $f$ is strictly monotonically increasing. Conversely,
	if $f^\prime(x)<0$, then $f$ is strictly monotonically decreasing.
\end{thm}

\begin{proof}
	Of \pref{theorem}{thm-differentiable-strictly-monotone}.
	\begin{flushleft}
		\gls{wlog}, take any $a<x<y<b$. By \pref{theorem}{thm-mean-value-theorem},
		there is a point $c\in[x,y]$ such that
		\begin{equation}
			\frac{f^\prime(x)}{g^\prime(x)}=\frac{f(b)-f(a)}{g(b)-g(a)}
		\end{equation}
		Since $f^\prime(c)>0$ it follows that $f(y)>f(x)$. Hence, by
		\pref{definition}{def-monotonicity}, $f$ is strictly monotonically increasing.
	\end{flushleft}
\end{proof}

\begin{rem}\label{rem-differentiable-strictly-monotone}
	The converse of \pref{theorem}{thm-differentiable-strictly-monotone} is false,
	\textit{i.e.} if $f$ is strictly monotonically increasing or decreasing, then its
	first derivative is either strictly positive or negative, respectively. Take for
	instance $f(x)=x^3$. This function is strictly increasing on the entire real line,
	but its derivative at $x=0$ is zero. But what we can say is that
	\begin{equation}\label{eq-differentiable-strictly-monotone}
		\left(f\text{ is differentiable } \implies f^\prime\geq0\right)
		\iff f\text{ is monotonically increasing}
	\end{equation}
\end{rem}

\begin{exm}\label{exm-differentiable-strictly-monotone}
	Let $f(x)=\arctan(x)$. This function is strictly increasing on $\mathbb{R}$,
	since $f^\prime(x)=\tfrac{1}{1+x^2}>0$ for all $x\in\mathbb{R}$. Next consider
	the function $g(x)=\tfrac{1}{x}$. Then $g^\prime(x)=-\tfrac{1}{x^2}<0$ for any
	$x\neq0$, so this function is strictly monotonically decreasing.
\end{exm}
