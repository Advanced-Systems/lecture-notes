\subsection{Sets}\label{subsec-sets}

% ==============================================================================
% ==============================================================================
% ==============================================================================

TODO: Add description

\subsubsection{Indicator Functions}\label{subsubsec-indicator-functions}

\begin{definition}\label{def-indicator-function}
    Let $X$ be a set. For each subset $A \subseteq X$ there is defined the 
    indicator function\footnote{Sometimes, the term characteristic function is 
    used to describe the function that indicates membership in a set in other 
    fields of mathematics. The term indicator function is more prominently used 
    in probability theory.} $\chi_A:X\rightarrow \{0,1\}$ of the set $A$ as
    \begin{equation}
        \chi_A(x) = \begin{cases}
            1,\;\text{ if } x \in A \\
            0,\;\text{ if } x \notin A
        \end{cases}
    \end{equation}
    We will write $\chi_A$ instead of $\chi_A(x)$ if the parameter
    $x$ is unambiguous.
\end{definition}

\begin{thm}\label{thm-complement-indicator-function}
    Let $\chi_A$ be an indicator function. Then the complement of $A$ (i.e. 
    $\overline{A} \defines A^C$) is
    \begin{equation*}
        \chi_{\overline{A}} = 1 - \chi_A
    \end{equation*}
\end{thm}

\begin{proof}
    Of \pref{theorem}{thm-complement-indicator-function}.
    \begin{align*}
        \chi_{\overline{A}}(x) &= \begin{cases}
            1,\;\text{ if } \neg(x \in A) \\
            0,\;\text{ if } \neg(x \notin A)
        \end{cases} \\
        &= \begin{cases}
            1-0,\;\text{ if } x \notin A \\
            1-1,\;\text{ if } x \in A
        \end{cases} \\
        &=1 - \begin{cases}
            0,\;\text{ if } x \notin A \\
            1,\;\text{ if } x \in A
        \end{cases} \\
        &= 1 - \chi_A(x)
    \end{align*}
\end{proof}

\begin{thm}\label{thm-set-cardinality-indicator-function}
    Let $\chi_A$ be an indicator function with $A$ as a finite set. Then,
    \begin{equation*}
        \abs{A} = \sum_{x \in X} \chi_A(x)
    \end{equation*}
\end{thm}

\begin{proof}
    Of theorem (\ref{thm-set-cardinality-indicator-function}).
    \begin{align*}
        \abs{A} &=\sum_{x\in A}\underbrace{\chi_A(x)}_{1}\\
                &=\sum_{x\in A}\chi_A(x)+\sum_{x\in X\setminus A}\underbrace{\chi_A(x)}_{0}\\
                &=\sum_{x\in (A\cup (X\setminus A))}\chi_A(x)\\
                &\overset{(\star)}{=}\sum_{x\in X}\chi_A(x).
    \end{align*}
    since 
    \begin{equation*}
        (\star) :\Leftrightarrow A\cup(X\setminus A)=(A\cup X)\setminus(A\setminus A)=X\setminus \emptyset=X
    \end{equation*}
\end{proof}

\begin{thm}\label{thm-cap-indicator-function}
    Let $\chi_A,\chi_B$ be indicator functions. Then:
    \begin{equation*}
        \chi_{A \cap B} = \chi_A\chi_B
    \end{equation*}
\end{thm}

\begin{proof}
    Of \pref{theorem}{thm-cap-indicator-function}.
    \begin{align*}
        \chi_{A \cap B} &= \begin{cases}
            1\;\text{ if } x \in (A \cap B)\\
            0\;\text{ if } x \notin (A \cap B)
        \end{cases}\\
        &= \begin{cases}
            1\;\text{ if } x \in A \land x \in B\\
            0\;\text{ if } x \notin A \land x \notin B
        \end{cases}\\
        &= \left(\begin{cases}
            1\;\text{ if } x \in A\\
            0\;\text{ if } x \notin A
        \end{cases}\right)\left(\begin{cases}
            1\;\text{ if } x \in B\\
            0\;\text{ if } x \notin B
        \end{cases}\right)\\
        &= \chi_A\chi_B
    \end{align*}
\end{proof}

\begin{thm}\label{thm-cup-indicator-function}
    Let $\chi_A,\chi_B$ be indicator functions. Then:
    \begin{equation*}
        \chi_{A \cup B} = \chi_A + \chi_B - \chi_A\chi_B
    \end{equation*}
\end{thm}

\begin{proof}
    Of \pref{theorem}{thm-cup-indicator-function}.
    By using the fact that $\overline{\overline{A}}=A$ in combination with 
    \pref{theorem}{thm-complement-indicator-function} and
    \pref{theorem}{thm-cap-indicator-function} we get that
    \begin{align*}
        \chi_{\overline{\overline{A \cup B}}} 
        &= 1 - \chi_{\overline{A \cup B}}\\
        &\overset{(\star)}{=} 1 - \chi_{\overline{A} \cap \overline{B}}\\
        &= 1 - \chi_{\overline{A}} \chi_{\overline{B}}\\
        &= 1 - (1 - \chi_A)(1 - \chi_B)\\
        &= 1 - \left(1 - \chi_B - \chi_A + \chi_A\chi_B\right)\\
        &= \chi_A - \chi_B + \chi_A\chi_B\\
    \end{align*}
    Recall that $(\star)$ uses one of De Morgan's laws which states that
    \begin{align*}
        (\star) &\iff \overline{A \cup B}\\
                &\iff \bigwedge_{x\in X}\{x\notin (A\cup B)\}\\
                &\iff \bigwedge_{x\in X}\{x\notin A \land x\notin B\}\\
                &\iff \overline{A} \cap \overline{B}
    \end{align*}
\end{proof}