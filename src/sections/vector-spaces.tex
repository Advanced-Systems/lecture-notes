\subsection{Vector Spaces}\label{subsec-vector-spaces}

\begin{flushleft}
	From here on, we will drop the arrows above all vectors if we discuss them
	in a more general sense. The previous notation often indicates vectors in
	two- or three-dimensional vector spaces and will still be used like that in
	examples to follow.
\end{flushleft}

\begin{definition}\label{def-vector-space}
	A vector space\footnote{Sometimes also called linear spaces.}
	$(\mathcal{V},+,\cdot)$ is a set $\mathcal{V}$ over a field $\mathcal{F}$
	that defines the following two operations \footnote{An element in $\mathcal{V}$
		is called a vector, and an element in $\mathcal{F}$ is called a scalar.}:
	\begin{align}
		+     & :\mathcal{V}\times\mathcal{V}\rightarrow\mathcal{V},\quad(v,w)\mapsto v+w\quad\text{(Addition)}\label{def-vector-addition}                                           \\
		\cdot & :\mathcal{F}\times\mathcal{V}\rightarrow\mathcal{V},\quad(\lambda,v)\mapsto\lambda\cdot v\quad\text{(Scalar Multiplication)}\label{def-vector-scalar-multiplication}
	\end{align}
	which in turn satisfy all rules listed down below
	\cite[p.119]{liesenMehrmann2015}:
	\begin{enumerate}
		\item[(V1)] $(\mathcal{V},+)$ is an abelian group
		\item[(V2)] Furthermore, for all $v,w\in\mathcal{V}$ and $\lambda,\mu\in\mathcal{F}$ apply the axioms:
			\begin{enumerate}
				\item[(a)] $\lambda\cdot v\in\mathcal{V}$
				\item[(b)] $1\cdot v=v$
				\item[(c)] $\lambda\cdot(v+w)=\lambda\cdot v +\lambda\cdot w$
				\item[(d)] $(\lambda+\mu)\cdot v=\lambda\cdot v +\mu\cdot v$
				\item[(e)] $(\lambda\cdot\mu)\cdot v=\lambda\cdot(\mu\cdot v)$
			\end{enumerate}
	\end{enumerate}
\end{definition}

\begin{exm}
	See below a short list of common vector spaces:
	\begin{enumerate}
		\item $\mathbb{R}^n$
		\item $\mathcal{M}_{m \times n}(\mathbb{R})$
		\item $\mathcal{F}^n$
		\item $\mathcal{M}_{m \times n}(\mathcal{F})$
		\item $\mathcal{F}[x]$
		\item $\{f:\mathbb{R}\rightarrow\mathbb{R}\}$
		\item The set of all continuos functions
		\item The set of all differentiable functions
	\end{enumerate}
\end{exm}

\begin{thm}\label{thm-zero-vector}
	Let $\mathcal{V}$ be a vector space over $\mathcal{F}$. Then,
	for all $ v \in\mathcal{V}$ and $\lambda\in\mathcal{F}$:
	\begin{enumerate}
		\item $0 \cdot v = 0$
		\item $\lambda\cdot 0 = 0$
		\item $\lambda\cdot v = 0 \implies \lambda=0 \lor v=0$
	\end{enumerate}
\end{thm}

\begin{definition}\label{def-vector-subspace}
	Let $\mathcal{V}$ be a vector space and $\mathcal{W}$ a subset of $\mathcal{V}$,
	i.e. $\mathcal{W}\subseteq\mathcal{V}$. Then $\mathcal{W}$ is also a vector
	subspace if $\mathcal{W}$ is a vector space with respect to the operations
	from $\mathcal{V}$.
\end{definition}

\begin{exm}\label{exm-vector-subspaces}
	Consider the situations described below as examples of vector subspaces:
	\begin{enumerate}
		\item Let $\mathcal{W}=\mathcal{F}[x]_{\leq n}$ and $\mathcal{V}=\mathcal{F}[x]$.
		\item Let $\mathcal{W}$ be the set of all functions subject to the restriction
		      that $f(1)=0$, and let $\mathcal{V}=\{f:\mathbb{R}\rightarrow\mathbb{R}\}$.
		\item The trivial subset $\mathcal{W}=\mathcal{V}$.
		\item Let $\mathcal{W}=\{0\}$, and $\mathcal{V}=\mathcal{F}^n$.
		\item Let $W=\left\{x\in\mathbb{R}\setbuild\begin{pmatrix}x\\0\\0\end{pmatrix}\right\}$
		      and $\mathcal{V}=\mathbb{R}^3$.
		\item Let $W=\left\{x,y\in\mathbb{R}\setbuild\begin{pmatrix}x\\y\\0\end{pmatrix}\right\}$
		      and $\mathcal{V}=\mathbb{R}^3$.
	\end{enumerate}
	In contrast to the previous examples, the list below describes situations in
	which $\mathcal{W}$ is not a vector subspace:
	\begin{enumerate}
		\item Let $\mathcal{W}=\mathcal{F}[x]_n$ and $\mathcal{V}=\mathcal{F}[x]$.
		\item Let $\mathcal{W}=\mathbb{R}^2$ and $\mathcal{V}=\mathbb{R}^3$
		\item Let $W=\left\{x,y,z\in\mathbb{Z}\setbuild\begin{pmatrix}x\\y\\z\end{pmatrix}\right\}$
		      and $\mathcal{V}=\mathbb{R}^3$.
	\end{enumerate}
	In the first example of the second list, $\mathcal{W}$ is not a vector subspace
	because it is not closed under addition. As for the second example, $\mathcal{W}$
	is not even a subset of $\mathcal{V}$. Finally, in the third example $\mathcal{W}$
	is not closed under scalar multiplication (since its domain is limited to the
	integers which is also the reason why $\mathbb{Z}$ is not a field).
\end{exm}

\begin{exm}\label{exm-vector-subspace}
	Let $\mathcal{W}=\left\{\begin{pmatrix}x\\y\\z\end{pmatrix}\setbuild x+2y-z=0\right\}$
	and $\mathcal{V}=\mathbb{R}^3$, so $\mathcal{W}\subseteq\mathbb{R}^3$. Then,
	\begin{itemize}
		\item $\begin{pmatrix}0\\0\\0\end{pmatrix}\in\mathcal{W}$
		\item $\begin{pmatrix}x\\y\\z\end{pmatrix}+\begin{pmatrix}x^\prime\\y^\prime\\z^\prime\end{pmatrix}=\begin{pmatrix}x+x^\prime\\y+y^\prime\\z+z^\prime\end{pmatrix}=[(x+x^\prime)+2(y+y^\prime)-(z+z^\prime)]\in\mathcal{W}$
		\item $\lambda\begin{pmatrix}x\\y\\z\end{pmatrix}=\begin{pmatrix}\lambda x\\ \lambda y\\ \lambda z\end{pmatrix}=\lambda x + 2 \lambda y - \lambda z = \lambda(x+2y-x)\in\mathcal{W}$
	\end{itemize}
	All the other conditions follow for free with the next theorem, but feel free
	complete the proof without this information.
\end{exm}

\subsubsection{Vector Subspaces}\label{subsubsec-vector-subspaces}

\begin{thm}\label{thm-supspace}
	Let $\mathcal{W}$ be a non-empty subset of $\mathcal{V}$ where $\mathcal{V}$
	is a vector space. Then $\mathcal{W}$ is also a vector subspace if and only
	if $\mathcal{W}$ is closed under addition and multiplication by a scalar.
\end{thm}

\begin{proof}
	Of \pref{theorem}{thm-supspace}.
	\begin{flushleft}
		\proofright: If $\mathcal{W}$ is already a
		vector subspace then there is nothing left to prove.
	\end{flushleft}
	\begin{flushleft}
		\proofleft: For any vector $w\in\mathcal{W}$ we know that
		$0=0\cdot w \in\mathcal{W}$ since $w$ is closed under scalar
		multiplication. For the same reasons, $-w=-1\cdot w \in\mathcal{W}$.
		The remaining axioms follow automatically from the ambient vector space
		$\mathcal{V}$.
	\end{flushleft}
\end{proof}

\begin{thm}\label{thm-homogeneous-solution-subspace}
	The solutions of a homogeneous system of linear equations is a subspace.
	\footnote{The solutions of an inhomogeneous system are not a subspace, since
		by definition 0 is not a solution to this kind of a system.}
\end{thm}

\begin{proof}
	Of \pref{theorem}{thm-homogeneous-solution-subspace}.
	\begin{flushleft}
		Let $Ax=0$ and $\mathcal{W}$ be the set of solutions that solves this
		homogeneous system of linear equations.
		\begin{enumerate}
			\item[(i)] As noted earlier, in a homogeneous there always exists
				the trivial solution $0$, therefore $0\in\mathcal{W}$.
			\item[(ii)] Let $x_1, x_2\in\mathcal{W}$. That immediately implies
				that $x_1+x_2\in\mathcal{W}$ because $A(x_1+x_2)=Ax_1+Ax_2=0+0=0\in\mathcal{W}$
			\item[(iii)] Let $x\in\mathcal{W},\lambda\in\mathbb{R}$. Then
				$A(\lambda x)=\lambda(Ax)=0\implies\lambda\cdot0=0\in\mathcal{W}$.
		\end{enumerate}
	\end{flushleft}
\end{proof}

\begin{exm}
	Let $\mathcal{V}=\mathcal{M}_{3 \times 3}(\mathbb{R})$ be the ambient vector
	space. Now suppose we want to create a subspace that contains the matrix
	$I_3=\inlinematrix{1&0&0\\0&1&0\\0&0&1}$.
	Then we'd also have to include
	\begin{itemize}
		\item $\inlinematrix{0&0&0\\0&0&0\\0&0&0}$ as additive identity matrix
		\item $\lambda\inlinematrix{1&0&0\\0&1&0\\0&0&1}$ for some $\lambda\in\mathbb{R}$, which will also
		      ensure closure under addition
	\end{itemize}
	To summarize, the smallest possible subspace that includes $I_3$ would then be
	\begin{equation*}
		\mathcal{W}=\left\{\lambda\in\mathbb{R}\setbuild\begin{pmatrix}
			\lambda & 0       & 0       \\
			0       & \lambda & 0       \\
			0       & 0       & \lambda
		\end{pmatrix}\right\}
	\end{equation*}
	Now suppose we want to include an another matrix as well, namely
	$\inlinematrix{0&0&1\\0&1&0\\1&0&0}$.
	Then the new subspace would look like
	\begin{equation*}
		\mathcal{U}=\left\{\lambda,\mu\in\mathbb{R}\setbuild\lambda\begin{pmatrix}
			1 & 0 & 0 \\
			0 & 1 & 0 \\
			0 & 0 & 1
		\end{pmatrix}+\mu\begin{pmatrix}
			0 & 0 & 1 \\
			0 & 1 & 0 \\
			1 & 0 & 0
		\end{pmatrix}\right\}
	\end{equation*}
\end{exm}

\subsubsection{Intersection of Vector Subspaces}\label{subsubsec-intersection-vector-subspaces}

\begin{thm}\label{thm-intersection-vector-subspace}
	If $\mathcal{U}$ and $\mathcal{W}$ are vector subspaces of $\mathcal{V}$, then
	so is their intersection $\mathcal{U}\cap\mathcal{W}$.
\end{thm}

\begin{proof}
	Of \pref{theorem}{thm-intersection-vector-subspace}.
	\begin{flushleft}
		We use \pref{theorem}{thm-supspace} to perform this proof:
		\begin{enumerate}
			\item[(i)] Since $\mathcal{U}$ and $\mathcal{W}$ are vector subspaces
				we know that $0\in\mathcal{U}$ and $0\in\mathcal{W}$. Therefore,
				$0\in\mathcal{U}\cap\mathcal{W}$.
			\item[(ii)] Let $v_1,v_2\in\mathcal{U}\cap\mathcal{W}$. Then
				$v_1,v_2\in\mathcal{U}\implies(v_1+v_2)\in\mathcal{U}$ since $\mathcal{U}$
				is a vector subspace. For the exact same reason this sum is also going
				to be in $\mathcal{W}$, thus $(v_1+v_2)\in\mathcal{U}\cap\mathcal{W}$.
			\item[(iii)] Let $v\in\mathcal{U}\cap\mathcal{W}$ and $\lambda\in\mathcal{F}$.
				Then $\lambda v\in\mathcal{U}$ and $\lambda v\in\mathcal{W}$, which in
				turn implies that $\lambda v\in\mathcal{U}\cap\mathcal{W}$.
		\end{enumerate}
	\end{flushleft}
\end{proof}

\begin{exm}\label{exm-intersection-union-subspace}
	Building on the idea of \pref{theorem}{thm-intersection-vector-subspace},
	could $\mathcal{U}\cup\mathcal{W}$ be also a candidate for a vector subspace?
	Suppose we have $\mathcal{V}=\mathbb{R}^2$ as an ambient vector space and
	$\mathcal{U}=\left\{\lambda\in\mathbb{R}\setbuild\inlinematrix{\lambda\\0}\right\}$
	together with
	$\mathcal{W}=\left\{\mu\in\mathbb{R}\setbuild\inlinematrix{0\\\mu}\right\}$.
	Then $\mathcal{U}\cap\mathcal{W}=\left\{\inlinematrix{0\\0}\right\}$ is a
	vector subspace, but $\mathcal{U}\cup\mathcal{W}$ is not a vector subspace.
	For example, the sum
	$\inlinematrix{1\\0}+\inlinematrix{0\\1}=\inlinematrix{1\\1}\notin\mathcal{U}\cup\mathcal{W}$
	does not belong to the union even though both terms of the sum belong to
	$\mathcal{U}$ and $\mathcal{W}$, respectively. So in general it is closure
	under addition that fails this suspected theorem.
\end{exm}

\begin{definition}\label{def-sum-of-subsets}
	Let $\mathcal{U},\mathcal{W}\subseteq\mathcal{V}$ be subsets of the ambient
	vector space. Then we define the sum of both these subsets as
	\begin{equation}
		\mathcal{U}+\mathcal{W}=\left\{
		u\in\mathcal{U},w\in\mathcal{W}\setbuild u+w
		\right\}
	\end{equation}
\end{definition}

\subsubsection{Sum of Vector Subspaces}\label{subsubsec-sum-subspace}

\begin{thm}\label{thm-sum-subspace}
	If $\mathcal{U},\mathcal{W}$ are subspaces of $\mathcal{V}$, then so is their
	sum.
\end{thm}

\begin{proof}
	Of \pref{theorem}{thm-sum-subspace}.
	\begin{flushleft}
		We use \pref{theorem}{thm-supspace}
		again prove this theorem. Let $u\in\mathcal{U},w\in\mathcal{W}$ and
		$\lambda\in\mathcal{F}$. Then,
		\begin{enumerate}
			\item[(i)] The additive identity $0$ is contained in this sum since
				\begin{equation*}
					0_\mathcal{U}+0_\mathcal{W}=0\in\mathcal{U}+\mathcal{W}
				\end{equation*}
			\item[(ii)] Since $\mathcal{V}$ is a vector space we can use the fact that
				\begin{equation*}
					\lambda(u+w)=\lambda u + \lambda w \in\mathcal{U}+\mathcal{W}
				\end{equation*}
			\item[(iii)] Let $u_1,u_2\in\mathcal{U}$ and $w_1,w_2\in\mathcal{W}$. Then,
				\begin{equation*}
					(u_1+w_1)+(u_2+w_2)=\left(\underbrace{(u_1+u_2)}_{\in\mathcal{U}}+
					\underbrace{(w_1+w_2)}_{\in\mathcal{W}}
					\right)\in\mathcal{U}+\mathcal{W}
				\end{equation*}
		\end{enumerate}
	\end{flushleft}
\end{proof}

\begin{exm}
	Take $\mathcal{U}$ and $\mathcal{W}$ from \pref{example}{exm-intersection-union-subspace}.
	Then the sum
	\begin{equation*}
		\mathcal{U}+\mathcal{W}=\left\{
		\lambda,\mu\in\mathbb{R}\setbuild
		\begin{pmatrix}
			\lambda \\0
		\end{pmatrix}+
		\begin{pmatrix}
			0 \\\mu
		\end{pmatrix}
		\right\}=\mathbb{R}^2=\mathcal{V}
	\end{equation*}
	is also a vector subspace.
\end{exm}

\begin{exm}
	Let $\mathcal{V}=\mathbb{R}^3$ the ambient vector space and
	$\mathcal{U}=\left\{\lambda\in\mathbb{R}\setbuild\inlinematrix{\lambda\\0\\0}\right\}$
	and
	$\mathcal{W}=\left\{\mu\in\mathbb{R}\setbuild\inlinematrix{0\\\mu\\0}\right\}$.
	Then their sum
	\begin{equation*}
		\mathcal{U}+\mathcal{W}=\left\{
		\lambda,\mu\in\mathbb{R}\setbuild
		\begin{pmatrix}
			\lambda \\\mu\\0
		\end{pmatrix}
		\right\}
	\end{equation*}
	is also a vector subspace.
\end{exm}

\subsubsection{Direct Sum}\label{subsubsec-direct-sum}

\begin{definition}\label{def-direct-sum}
	We call $\mathcal{U}\oplus\mathcal{W}$ a direct sum if any element in
	$\mathcal{U}+\mathcal{W}$ can be expressed uniquely as $u+w$ where
	$u\in\mathcal{U}$ and $w\in\mathcal{W}$.
\end{definition}

\begin{exm}
	Here are two examples to illustrate the point made in \pref{definition}{def-direct-sum}:
	\begin{enumerate}
		\item \textbf{Example for a direct sum}:
		      Let
		      $\mathcal{W}=\left\{a,b\in\mathbb{R}\setbuild\inlinematrix{a\\b\\0}\right\}$
		      and
		      $\mathcal{U}=\left\{c\in\mathbb{R}\setbuild\inlinematrix{0\\0\\c}\right\}$.
		      Then
		      $\mathcal{U}+\mathcal{W}=\left\{a,b,c\in\mathbb{R}\setbuild\inlinematrix{a\\b\\c}\right\}=\mathbb{R}^3$
		      and
		      \begin{equation*}
			      \begin{pmatrix}
				      a \\b\\0
			      \end{pmatrix}+
			      \begin{pmatrix}
				      0 \\0\\c
			      \end{pmatrix}=
			      \begin{pmatrix}
				      a \\b\\c
			      \end{pmatrix}\in\mathcal{U}+\mathcal{W}
		      \end{equation*}
		      is the only non-trivial representation for a vector in this vector space.
		      The trivial case is implied in the intersection
		      \begin{equation*}
			      \mathcal{U}\cap\mathcal{W}=\left\{\begin{pmatrix}
				      0 \\0\\0
			      \end{pmatrix}\right\}
		      \end{equation*}
		      and therefore is the sum of both vector spaces by
		      definition (\ref{def-direct-sum}) called a \textit{direct sum}.
		\item \textbf{Example for an indirect sum}:
		      Let
		      $\mathcal{U}=\left\{a,b\in\mathbb{R}\setbuild\inlinematrix{a\\b\\0}\right\}$
		      and
		      $\mathcal{W}=\left\{c,d\in\mathbb{R}\setbuild\inlinematrix{0\\c\\d}\right\}$.
		      Then
		      $\mathcal{U}+\mathcal{W}=\left\{a,b,c\in\mathbb{R}\setbuild\inlinematrix{a\\b\\c}\right\}=\mathbb{R}^3$
		      and
		      \begin{equation*}
			      \begin{pmatrix}
				      a \\b\\0
			      \end{pmatrix}+
			      \begin{pmatrix}
				      0 \\c\\d
			      \end{pmatrix}=
			      \begin{pmatrix}
				      a \\b+c\\d
			      \end{pmatrix}\in\mathcal{U}+\mathcal{W}
		      \end{equation*}
		      But this time there is no longer a unique way to represent a vector from
		      this sum, e.g.
		      \begin{equation*}
			      \begin{pmatrix}
				      1 \\2\\3
			      \end{pmatrix}=
			      \begin{pmatrix}
				      1 \\2\\0
			      \end{pmatrix}+
			      \begin{pmatrix}
				      0 \\0\\3
			      \end{pmatrix}=
			      \begin{pmatrix}
				      1 \\0\\0
			      \end{pmatrix}+
			      \begin{pmatrix}
				      0 \\2\\3
			      \end{pmatrix}
		      \end{equation*}
		      since
		      \begin{equation*}
			      \mathcal{U}\cap\mathcal{W}=\left\{e\in\mathbb{R}\setbuild\begin{pmatrix}
				      0 \\e\\0
			      \end{pmatrix}\right\}
		      \end{equation*}
		      contains more than the trivial intersection
		      $\inlinematrix{0\\0\\0}$. Hence, this
		      is not a \textit{direct sum}.
	\end{enumerate}
\end{exm}

\begin{thm}\label{thm-direct-sum}
	Let $\mathcal{U},\mathcal{W}\subseteq\mathcal{V}$ be subspaces. Then,
	\begin{equation}
		\mathcal{U}\oplus\mathcal{W}\Leftrightarrow
		\mathcal{U}\cap\mathcal{W}=
		\left\{\inlinematrix{0\\0\\0}\right\}
	\end{equation}
\end{thm}

\begin{proof}
	Of \pref{theorem}{thm-direct-sum}:
	\begin{flushleft}
		\proofright: Let $v\in\mathcal{W}\cap\mathcal{W}$.
		Then,
		\begin{equation*}
			v=\underbrace{v}_{\in\mathcal{U}}+\underbrace{0}_{\in\mathcal{W}}
			=\underbrace{0}_{\in\mathcal{U}}+\underbrace{v}_{\in\mathcal{W}}
		\end{equation*}
		but since there is supposed to a unique representation of $v$ then
		$v=0$ and $0=v$.
	\end{flushleft}
	\begin{flushleft}
		\proofleft: Now let $\mathcal{U}$ and $\mathcal{W}$ only have a trivial
		intersection, i.e. $\mathcal{U}\cap\mathcal{W}=\{0\}$. Then take an arbitrary
		element $v\in\mathcal{U}+\mathcal{W}$ and write
		\begin{equation*}
			v=u_1+w_1=u_2+w_2
		\end{equation*}
		We show that $u_1=u_2$ and $w_1=w_2$ by manipulating the equation to
		\begin{equation*}
			\underbrace{u_1-u_2}_{\in\mathcal{U}}=\underbrace{w_2-w_1}_{\in\mathcal{W}}=0
		\end{equation*}
		since $\mathcal{U}$ and $\mathcal{W}$ come both with closure under addition
		and the intersection was assumed to be trivial from the very start, so
		$u_1=u_2$ and $w_1=w_2$, therefore we can write $\mathcal{U}\oplus\mathcal{W}$.
	\end{flushleft}
\end{proof}
