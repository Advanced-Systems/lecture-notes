\subsection{The Adjugate Matrix}\label{subsec-adjugate-matrices}

\begin{definition}
	Let $A$ be a $n \times n$ matrix. The so-called adjugate matrix $\fadj(A)$
	is defined as the $ij$'th entry of $\fadj(A)$ which is $(-1)^{i+j}M_{ji}$.
\end{definition}

\begin{exm}\label{exm-finding-adj}
	Let $A=\inlinematrix{2&3&-4\\0&4&2\\1&-1&5}$. We can find the adjugate matrix by
	\begin{equation*}
		\fadj(A)=\begin{pmatrix}
			\begin{vmatrix}
				-4 & 2 \\
				-1 & 5
			\end{vmatrix}  &
			-\begin{vmatrix}
				3  & -4 \\
				-1 & 5
			\end{vmatrix} &
			\begin{vmatrix}
				3  & -4 \\
				-4 & 2
			\end{vmatrix}    \\[12pt]
			-\begin{vmatrix}
				0 & 2 \\
				1 & 5
			\end{vmatrix} &
			\begin{vmatrix}
				2 & -4 \\
				1 & 5
			\end{vmatrix}  &
			-\begin{vmatrix}
				2 & -4 \\
				0 & 2
			\end{vmatrix}   \\[12pt]
			\begin{vmatrix}
				0 & -4 \\
				1 & -1
			\end{vmatrix}  &
			-\begin{vmatrix}
				2 & 3  \\
				1 & -1
			\end{vmatrix} &
			\begin{vmatrix}
				2 & 3  \\
				0 & -4
			\end{vmatrix}
		\end{pmatrix}=
		\begin{pmatrix}
			-18 & -11 & 10 \\
			2   & 14  & -4 \\
			4   & 5   & -8
		\end{pmatrix}
	\end{equation*}
\end{exm}

\begin{thm}\label{thm-adj-det-equation}
	Let $A$ be a square matrix. Then,
	\begin{equation}
		A\cdot\fadj(A)=\det(A)\cdot I = \begin{pmatrix}
			\abs{A} & 0       & 0       & \cdots & 0       \\
			0       & \abs{A} & 0       & \cdots & 0       \\
			0       & 0       & \abs{A} & \cdots & 0       \\
			\vdots  & \vdots  & \vdots  & \ddots & \vdots  \\
			0       & 0       & 0       & \cdots & \abs{A}\end{pmatrix}
	\end{equation}
\end{thm}

\begin{crl}\label{crl-inverse-adjugate-eq}
	If $A$ is invertible, then
	\begin{equation}
		A^{-1}=\frac{1}{\det(A)}\fadj(A)
	\end{equation}
\end{crl}

\begin{proof}
	Of \pref{corollary}{crl-inverse-adjugate-eq}.
	\begin{flushleft}
		If $A$ is invertible, then $\det(A)\neq0$. Therefore we can take the
		equation of \pref{theorem}{thm-adj-det-equation} and divide both sides
		by the determinant of $A$ which immediately yields the desired result after
		apply a series of transformations:
		\begin{align*}
			\det(A)\cdot I         & = A\cdot\fadj(A)                                    \\
			\implies I             & =\frac{1}{\det(A)}A\cdot\fadj(A)                    \\
			\implies A^{-1}\cdot I & =A^{-1}\left(\frac{1}{\det(A)}A\cdot\fadj(A)\right) \\
			\implies A^{-1}        & =\frac{1}{\det(A)}\left(A^{-1}A\right)\cdot\fadj(A) \\
			\implies A^{-1}        & =\frac{1}{\det(A)}\fadj(A)
		\end{align*}
	\end{flushleft}
\end{proof}

\begin{exm}
	Find the inverse of the matrix $A$ by using the method described in
	\pref{corollary}{crl-inverse-adjugate-eq} by using the results of
	\pref{example}{exm-finding-adj}.
	\begin{flushleft}
		\textbf{Answer}: What's missing to follow through this example is computing
		the determinant of $A$. We also use this as an opportunity to use statement 7
		of \pref{theorem}{thm-determinant-properties} in the second transformation
		by using the elementary matrix $L_{13}(-2)$:
		\begin{align*}
			\det(A) & =\begin{vmatrix}
				2 & 3  & -4 \\
				0 & 4  & 2  \\
				1 & -1 & 5
			\end{vmatrix} \\
			        & =\begin{vmatrix}
				0 & 5  & -14 \\
				0 & 4  & 2   \\
				1 & -1 & 5
			\end{vmatrix} \\
			        & =\begin{vmatrix}
				5 & -14 \\
				4 & -2
			\end{vmatrix} \\
			        & =-46
		\end{align*}
	\end{flushleft}
	Therefore, the inverse matrix is
	\begin{align*}
		A^{-1} & =-\frac{1}{46}\begin{pmatrix}
			-18 & -11 & 10 \\
			2   & 14  & -4 \\
			4   & 5   & -8
		\end{pmatrix} \\
		       & =\begin{pmatrix}
			\frac{18}{46} & \frac{11}{46}  & \frac{10}{46} \\[4pt]
			-\frac{2}{46} & -\frac{14}{46} & \frac{4}{46}  \\[4pt]
			-\frac{4}{46} & -\frac{5}{46}  & \frac{8}{46}
		\end{pmatrix}              \\
		       & =\begin{pmatrix}
			\frac{9}{23}  & \frac{11}{46} & \frac{5}{23} \\[4pt]
			-\frac{1}{23} & -\frac{7}{23} & \frac{2}{23} \\[4pt]
			-\frac{2}{23} & -\frac{5}{46} & \frac{4}{23}
		\end{pmatrix}
	\end{align*}
\end{exm}\

\begin{rem}\label{rem-adj-properties}
	Here are a couple of statement that can be easily verified by the reader:
	\begin{enumerate}
		\item $\fadj(I)=I$
		\item $\fadj(AB)=\fadj(B)\fadj(A)$
	\end{enumerate}
\end{rem}
