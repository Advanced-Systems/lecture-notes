\subsection{Logic}\label{subsec-logic}

% ==============================================================================
% ==============================================================================
% ==============================================================================

\subsubsection{Truth Tables}\label{subsubsec-truth-table}

Propositional logic is a formal mathematical framework that encodes the truth value
of one or more propositions. A proposition is a statement that, by itself, is either
true or false. These propositions can be combined with binary operations to determine
the truth value of a statement expressed in the vernacular. While every proposition
is a statement, not every statement is a valid proposition.

\begin{enumerate}
	\item Questions are invalid propositions.
	\item Statements where the intention is unclear or ambiguous don't qualify as propositions, either.
\end{enumerate}

In practice, the variables \(A,B,C,\dots\) are often used to denote propositions.
Therefore, they are also sometimes referred to as \emph{propositional variables}
that can take on one of two possible values: \(T\) (true) or \(F\) (false). In
computer science, \(1\) (true) or \(0\) (false) are commonly used instead. It is
worth adding that many programming languages interpret any number \(n\neq0\) as
true\footnote{Take Python or C++ as prominent examples for this behavior.}. While
propositional logic has its place, inference systems are used in formal reasoning
instead, though this topic is beyond the scope of this subsection.

\begin{definition}\label{def-axiom}
	An axiom is a statement that is taken to be true and is confined within the
	system of logic they define. They establish the premise of reasoning and cannot
	be formally proved.
\end{definition}

There are five important axioms from which one can obtain a series of useful logical
statements that will later be used to formulate mathematical sound proofs. In order
to make this material more accessible, each axiom will be prefaced by a quote written
in the vernacular that accurately reflects the statement in question.

\begin{displayquote}
	``\emph{It takes courage and strength to be empathetic.}''
	--- Jacinda Ardern\footnote{New Zealand politician who has been serving as the
		40th prime minister of New Zealand and leader of the Labour Party since 2017.}
\end{displayquote}

In this example, a person is said to be empathetic (\(C\)) if one is both
courageous (\(A\)) \emph{and} strong in character (\(B\)).

\begin{table}[hbt!]
	\centering
	\rowcolors{1}{}{lightgray}
	\begin{tabular}{*{6}{c}}
		$(A$ & $\land$ & $B)$ & $\rightarrow$ & $C$ \\
		T    &         & T    &               & T   \\
		T    &         & F    &               & T   \\
		F    &         & T    &               & F   \\
		F    &         & F    &               & F   \\
	\end{tabular}
	\caption{Conjunction}\label{table-conjunction}
\end{table}

\begin{displayquote}
	``\emph{So much of life is not about whether you're good or bad, or right or
		wrong, or can afford or not afford -- it's about timing.}''
	-- Adrian Anthony Gill\footnote{Scottish writer and critic.}
\end{displayquote}

In natural languages, a distinction is often drawn implicitly between inclusive
and exclusive disjunctions. See \pref{table}{table-disjunction} as an example of
the inclusive disjunction (\emph{or}). The truth table for the exclusive disjunction
(\emph{xor}) will be introduced later in \pref{subsection}{subsubsec-circuit-representation}.

\begin{table}[hbt!]
	\centering
	\rowcolors{1}{}{lightgray}
	\begin{tabular}{*{6}{c}}
		$(A$ & $\lor$ & $B)$ & $\rightarrow$ & $C$ \\
		T    &        & T    &               & T   \\
		T    &        & F    &               & T   \\
		F    &        & T    &               & T   \\
		F    &        & F    &               & F   \\
	\end{tabular}
	\caption{Disjunction}\label{table-disjunction}
\end{table}

\begin{displayquote}
	``\emph{'Awkward' implies both solidarity and implication. Nobody is exempt.}''
	-- Elif Batuman\footnote{American author, academic, and journalist.}
\end{displayquote}

One interesting thing to note about \pref{table}{table-subjunction} is that it is
possible to arrive to true statements even if the premise or assumption turned out
to be incorrect, an important implication of this being that there are no valuable
insights to be gained from a false premise. Therefore, it is of utmost importance
in logical reasoning to verify that the starting arguments have a positively affirmative
truth value.

\begin{table}[hbt!]
	\centering
	\rowcolors{1}{}{lightgray}
	\begin{tabular}{*{6}{c}}
		$(A$ & $\Rightarrow$ & $B)$ & $\rightarrow$ & $C$ \\
		T    &               & T    &               & T   \\
		T    &               & F    &               & F   \\
		F    &               & T    &               & T   \\
		F    &               & F    &               & T   \\
	\end{tabular}
	\caption{Subjunction}\label{table-subjunction}
\end{table}

\begin{displayquote}
	``\emph{One man is equivalent to all Creation. One man is a World in miniature.}''
	-- Albert Pike\footnote{American author, poet, orator, editor, lawyer, jurist,
		and prominent member of the Freemasons}
\end{displayquote}

Logically equivalent statements share the same truth value across all occurring
propositional variables. They become important later in ring proofs where multiple
distinct propositions implicate each other in all directions.

\begin{table}[hbt!]
	\centering
	\rowcolors{1}{}{lightgray}
	\begin{tabular}{*{6}{c}}
		$(A$ & $\Leftrightarrow$ & $B)$ & $\rightarrow$ & $C$ \\
		T    &                   & T    &               & T   \\
		T    &                   & F    &               & F   \\
		F    &                   & T    &               & F   \\
		F    &                   & F    &               & T   \\
	\end{tabular}
	\caption{Bisubjunction}\label{table-bisubjunction}
\end{table}

\begin{exm}\label{exm-logical-equivalent}
	It is left to the reader as an exercise to use a truth table to verify the
	following proposition:
	\begin{equation}
		\left((A \Rightarrow B) \land (B \Rightarrow A)\right) \Leftrightarrow (A \Leftrightarrow B)
	\end{equation}
\end{exm}

\begin{displayquote}
	``\emph{No worse fate can befall a young man or woman than becoming prematurely
		entrenched in prudence and negation.}''
	-- Knut Hamsun\footnote{Norwegian writer and Nobel Prize winner in Literature
		in 1920.}
\end{displayquote}

Negation is one of the simplest unary operations in that it inverts the truth
value of a propositional variable.

\begin{table}[hbt!]
	\centering
	\rowcolors{1}{}{lightgray}
	\begin{tabular}{*{6}{c}}
		$(\neg B)$ & $\rightarrow$ & $C$ \\
		T          &               & F   \\
		F          &               & T   \\
	\end{tabular}
	\caption{Negation}\label{table-negation}
\end{table}

% ==============================================================================
% ==============================================================================
% ==============================================================================

\subsubsection{Circuit Representation of Logical Operations}\label{subsubsec-circuit-representation}

Truth tables are very useful in determining the nature of logic gates.
In \pref{table}{subsubsec-truth-table} find defined the axioms which build the basis
for this section. Take \texttt{XOR}, for instance: this logic gate evaluates to $1$
if and only if one of both poles receives $1$ as an input as opposed to \texttt{OR}
which also accepts two positive-valued states favorably.

\begin{table}[hbt!]
	\centering
	\rowcolors{1}{}{lightgray}
	\begin{tabular}{*{6}{c}}
		$A$ & $B$ & $A\texttt{ OR }B$ & $A\texttt{ AND }B$ & $A\texttt{ NAND }B$ & $(A\texttt{ OR }B)\texttt{ AND }(A\texttt{ NAND }B)$ \\
		1   & 1   & 1                 & 1                  & 0                   & 0                                                    \\
		1   & 0   & 1                 & 0                  & 1                   & 1                                                    \\
		0   & 1   & 1                 & 0                  & 1                   & 1                                                    \\
		0   & 0   & 0                 & 0                  & 1                   & 0                                                    \\
	\end{tabular}
	\caption{\texttt{XOR} Truth Table}\label{truth-table-xor}
\end{table}

Based on this truth table it is possible to implement a logic gate that replicates
a \texttt{XOR} gate in the following way:

\begin{figure}[hbt!]
	\centering
	\begin{circuitikz}
		% gates
		\node[or port,draw] at (0,2) (or) {};
		\node[nand port,draw] at (0,0) (nand) {};
		\node[and port,draw] at (2,1) (and) {};
		% inputs
		\node at ($(or.in 1) + (-1,0)$) (A) {$A$};
		\node at ($(nand.in 2) + (-1,0)$) (B) {$B$};
		% wires
		\draw (A) -- (or.in 1);
		\draw (A) -- (nand.in 1);
		\draw (B) -- (or.in 2);
		\draw (B) -- (nand.in 2);
		\draw (or.out) -- (and.in 1);
		\draw (nand.out) -- (and.in 2);
		\draw (and.out) -- ($(and.out) + (0.5,0)$) node[anchor=west] (Q) {$Q$};
	\end{circuitikz}
	\caption{Circuit of an \texttt{XOR} gate}\label{circuit-xor-gate}
\end{figure}

A half adder is a circuit that adds two binary numbers, each one bit in size.
The result $S$ is also represented by a one bit value, so in case of
$1_2+1_2=(10)_2$ the second digit must be carried over in $C$ (hence the name
\textit{carrier}) in order to be preserved for future operations.

\begin{figure}[hbt!]
	\centering
	\begin{circuitikz}
		% xor gate
		\node[xor port,draw] at (0,2) (xor) {};
		\draw (xor.in 1) -- ($(xor.in 1) + (-0.75,0)$) node[anchor=east] (A) {$A$};
		\draw (xor.in 2) -- ($(xor.in 2) + (-0.75,0)$) node[anchor=east] (B) {$B$};
		\draw (xor.out) -- ($(xor.out) + (0.75,0)$) node[anchor=west] (S) {$S$};
		% and gate   
		\node[and port,draw] at (0,0) (and) {};
		\draw (xor.in 1) -- (and.in 1);
		\draw ($(xor.in 2) + (-0.25,0)$) -- ($(and.in 2) + (-0.25,0)$) -- (and.in 2);
		\draw (and.out) -- ($(and.out) + (0.75,0)$) node[anchor=west] (C) {$C$};
	\end{circuitikz}
	\caption{Circuit of an half adder}\label{circuit-half-adder}
\end{figure}

\begin{table}[hbt!]
	\centering
	\rowcolors{1}{}{lightgray}
	\begin{tabular}{*{4}{c}}
		$A$ & $B$ & $S$ & $C$ \\
		1   & 1   & 0   & 1   \\
		1   & 0   & 1   & 0   \\
		0   & 1   & 1   & 0   \\
		0   & 0   & 0   & 0   \\
	\end{tabular}
	\caption{Half Adder Truth Table}\label{truth-table-half-adder}
\end{table}

As opposed to an half adder, a full adder takes one more input (a so-called \texttt{carry in})
and two outputs, \textit{i.e.} \texttt{carry out} and \texttt{sum}. To build a
more sophisticated adder, chain half adders in a way that allows the result of
the previous sum to be carried over as \texttt{cin} to the next half adder.

\begin{figure}[hbt!]
	\centering
	\begin{circuitikz}
		% gates
		\node[xor port,draw,anchor=center] at (0,4) (xor1) {D};
		\node[xor port,draw] at (2,4) (xor2) {};
		\node[and port,draw] at (2,2) (and1) {E};
		\node[and port,draw] at (2,0) (and2) {F};
		\node[or port,draw] at (4,1) (or) {};
		% inputs
		\node at ($(xor1.in 1) + (-1,0)$) (A) {$A$};
		\node at ($(xor1.in 2) + (-1,0)$) (B) {$B$};
		\node at ($(A) + (0,-1.5)$) (C) {$C_{in}$};
		% wires
		\draw (A) -- (xor1.in 1);
		\draw (B) -- (xor1.in 2);
		\draw (xor1.out) -- (xor2.in 1);
		\draw (C) -- (xor2.in 2);
		\draw (C) -- (and1.in 1);
		\draw (xor1.out) -- ($(and1.in 2)-(0.5,0)$) -- (and1.in 2);
		\draw ($(A)+(0.75,0)$) -- ($(A)+(0.75,-4.56)$) -- (and2.in 2);
		\draw ($(B)+(1,0)$) -- ($(B)+(1,-3.425)$) -- (and2.in 1);
		\draw (and1.out) -- (or.in 1);
		\draw (and2.out) -- (or.in 2);
		\draw (xor2.out) -- ($(xor2.out) + (0.5,0)$) node[anchor=west] (S) {$S$};
		\draw (or.out) -- ($(or.out) + (0.5,0)$) node[anchor=west] (Q) {$C_{out}$};
	\end{circuitikz}
	\caption{Circuit of an full adder}\label{circuit-full-adder:1}
\end{figure}

Circuit \pref{figure}{circuit-full-adder:1} introduced three additional truth values
to make the truth \pref{table}{truth-table-full-adder} for the full adder easier
to read\footnote{Note that $S:\Leftrightarrow D\texttt{ XOR } C_{in}$ and
	$C_{out} :\Leftrightarrow E\texttt{ OR }F$}. It is helpful to think of gate $D$
and gate $F$ as the first half adder.

\begin{table}[hbt!]
	\centering
	\rowcolors{1}{}{lightgray}
	\begin{tabular}{*{8}{c}}
		$A$ & $B$ & $C_{in}$ & $A\texttt{ XOR }B$ & $D\texttt{ XOR }C_{in}$ & $D\texttt{ AND }C_{in}$ & $F$ & $E\texttt{ OR }F$ \\
		1   & 1   & 1        & 0                  & 1                       & 0                       & 1   & 1                 \\
		1   & 1   & 0        & 0                  & 0                       & 0                       & 1   & 1                 \\
		1   & 0   & 1        & 1                  & 0                       & 1                       & 0   & 1                 \\
		0   & 1   & 1        & 1                  & 0                       & 1                       & 0   & 1                 \\
		1   & 0   & 0        & 1                  & 1                       & 0                       & 0   & 0                 \\
		0   & 1   & 0        & 1                  & 1                       & 0                       & 0   & 0                 \\
		0   & 0   & 1        & 0                  & 1                       & 0                       & 0   & 0                 \\
		0   & 0   & 0        & 0                  & 0                       & 0                       & 0   & 0                 \\
	\end{tabular}
	\caption{Full Adder Truth Table}\label{truth-table-full-adder}
\end{table}
