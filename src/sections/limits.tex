\subsection{Limits}\label{subsec-limits}

\begin{definition}\label{def-epsilon-delta-definition-limit}
    Let $f(x)$ be a function with $a,b\in\mathbb{R}$ with $a$ as a limit point of $\domain{f}$.
    Then the the function $f$ has the limit $b$ as $x$ approaches $a$, \textit{i.e.}    
    \begin{equation}
        \bigwedge_{\varepsilon>0}\bigvee_{\delta>0}\bigwedge_{x\in\domain{f}} 
        \left(0<\abs{x-a}<\delta\implies\abs{f(x)-b}<\varepsilon\right)
    \end{equation}
    The limit of this function is denoted by
    \begin{equation}
        \lim_{x\to a}f(x)=b,
    \end{equation}
\end{definition}

\begin{rem}
    The equation $\displaystyle\lim_{x\to a}f(x)=b$ is equivalent to the following
    expressions:
    \begin{enumerate}
        \item For $f$ there exists a limit (point) at $a$.
        \item This limit of $f$ is equal to $b$.
    \end{enumerate}
    In other words, $f$ has a limit at $a$ if
    \begin{equation*}
        \bigvee_{b\in\mathbb{R}}\bigwedge_{\varepsilon>0}\bigvee_{\delta>0}\bigwedge_{x\in\domain{f}} 
        \left(0<\abs{x-a}<\delta \implies \abs{f(x)-b}<\varepsilon\right)
    \end{equation*}
\end{rem}

\begin{exm}\label{exm-epsilon-delta-definition-limit:1}
    Show that 
    \begin{equation*}
        \lim_{x\to3}\frac{x-1}{2}=1
    \end{equation*}
    by using the epsilon-delta definition for limits.
    \begin{flushleft}
        \textbf{Answer}: Let $\varepsilon>0$ and $b\in\mathbb{R}$. Take 
        $\delta\defines2\varepsilon$; Then $\abs{x-3}<\delta$, which implies
        \begin{align*}
            \abs{f(x)-b}&=\abs[\Bigg]{\frac{x-1}{2}-1}\\
                        &=\frac{\abs{x-3}}{2}\\
                        &<\frac{\delta}{2}\\
                        &=\varepsilon
        \end{align*}
    \end{flushleft}
\end{exm}

\begin{exm}\label{exm-epsilon-delta-definition-limit:2}
    Show that for some $a\in\mathbb{R}$,
    \begin{equation*}
        \lim_{x\to a}\sin(x)=\sin(a)
    \end{equation*}
    by using the epsilon-delta definition for limits.
    \begin{flushleft}
        \textbf{Answer}: Let $\varepsilon>0$ and $b\in\mathbb{R}$. Take 
        $\varepsilon\defines\delta$ such that $\abs{x-a}<\delta$. Further, recall that
        \begin{align*}
            &\text{(A)}:\forall\alpha\in\mathbb{R}:\abs{\cos(\alpha)}\leq1\\
            &\text{(B)}:\forall\alpha\in\mathbb{R}:\abs{\sin(\alpha)}\leq\abs{\alpha}
        \end{align*}
        \begin{align*}
            \abs{f(x)-b}&=\abs{\sin(x)-\sin(a)}\\
                        &=\abs[\Bigg]{2\sin\left(\frac{x-a}{2}\right)\cos\left(\frac{x+a}{2}\right)}\\
                        &=2\cdot\abs[\Bigg]{\sin\left(\frac{x-a}{2}\right)}\abs[\Bigg]{\cos\left(\frac{x+a}{2}\right)}\\
                        &\leq 2\cdot\frac{\abs{x-a}}{2} && \text{note (A) \& (B)}\\
                        &=\varepsilon
        \end{align*}
    \end{flushleft}
\end{exm}

\begin{exm}\label{exm-epsilon-delta-definition-limit:3}
    Show \cite[p.69]{wuest2009} that the function $f(x)=\tfrac{x^2-3x+2}{x(x-1)}$
    with $x\in\mathbb{R}\setminus\{0,1\}$ has the limit $b=-1$ as $x\to1$.
    \begin{flushleft}
        \textbf{Answer}: Let $\varepsilon>0$.
        \begin{align}
            \abs{f(x)-(-1)}&=\abs[\Bigg]{\frac{x^2-3x+2}{x(x-1)}+1}\nonumber\\
                           &=\abs[\Bigg]{\frac{(x-2)(x-1)}{x(x-1)}+1}\nonumber\\
                           &=\abs[\Bigg]{\frac{x-2}{x}+1}\nonumber\\
                           &=\abs[\Bigg]{\frac{2x-2}{x}}\nonumber\\
                           &=\frac{2}{\abs{x}}\cdot\abs{x-1}\label{eq3-epsilon-delta-definition-limit:1}
        \end{align}
        Since we aim for the neighborhood of $x=1$ we can impose the following
        restriction on this inequality:
        \begin{equation}\label{eq3-epsilon-delta-definition-limit:2}
            0<\abs{x-1}<\frac{1}{2}
        \end{equation}
        This in turn implies
        \begin{align}
            \abs{x}&=\abs{x-1+1}\nonumber\\
                   &\geq1-\abs{x-1}\nonumber\\
                   &>1-\frac{1}{2} && \text{\pref{equation}{eq3-epsilon-delta-definition-limit:2}}\nonumber\\
                   &=\frac{1}{2}\label{eq3-epsilon-delta-definition-limit:3}
        \end{align}
        Hence,
        \begin{align}
            \abs{f(x)-b}&=\frac{2}{\abs{x}}\cdot\abs{x-1} && \text{\pref{equation}{eq3-epsilon-delta-definition-limit:1}}\nonumber\\
                        &<\frac{2}{\tfrac{1}{2}}\cdot\abs{x-1} && \text{\pref{equation}{eq3-epsilon-delta-definition-limit:3}}\nonumber\\
                        &=4\cdot\abs{x-1}\label{eq3-epsilon-delta-definition-limit:4}
        \end{align}
    \end{flushleft}
    Let $q(\varepsilon)\defines\min\left\{\tfrac{\varepsilon}{4},\tfrac{1}{2}\right\}$.
    Then for all $\varepsilon>0$ there exists a $\delta\defines q(\varepsilon)$ such that
    for $x\in\domain{f}$ it follows that
    \begin{align*}
        0<\abs{x-1}<\delta \implies \abs{f(x)-(-1)}&<4\cdot\abs{x-1}&&\text{\pref{equation}{eq3-epsilon-delta-definition-limit:4}}\\
                                                   &<4\cdot\frac{\varepsilon}{4}\\
                                                   &=\varepsilon
    \end{align*}
\end{exm}

\begin{exm}\label{exm-epsilon-delta-definition-limit:4}
    Show that the function $f(x)=\tfrac{x^2-4x+3}{2x-6}$ with $x\in\mathbb{R}\setminus\{3\}$
    has the limit $b=1$ as $x\to3$.
    \begin{flushleft}
        \textbf{Answer}: Let $\varepsilon>0$. Define $\delta\defines2\varepsilon$,
        such that $\abs{x-3}<\delta$, wherefore
        \begin{align*}
            \abs{f(x)-b}&=\abs[\Bigg]{\frac{x^2-4x+3}{2x-6}-1}\\
                        &=\abs[\Bigg]{\frac{(x-1)(x-3)}{2(x-3)}-1}\\
                        &=\abs[\Bigg]{\frac{1}{2}(x-1)-1}\\
                        &=\abs[\Bigg]{\frac{1}{2}x-\frac{3}{2}}\\
                        &=\frac{1}{2}\abs{x-3}\\
                        &<\frac{1}{2}\cdot2\varepsilon\\
                        &=\varepsilon
        \end{align*}
    \end{flushleft}
\end{exm}

\begin{exm}\label{exm-epsilon-delta-definition-limit:5}
    Show that for $a>0$ ($a\in\mathbb{R}$),
    \begin{equation*}
        \sqrt{x}\tolim{x}{a}\sqrt{a}
    \end{equation*}
    \begin{flushleft}
        \textbf{Answer}: Let $\varepsilon>0$. When we define $\delta\defines\min\left\{a,\varepsilon\sqrt{a}\right\}$,
        then $\abs{x-a}<\delta$ implies that
        \begin{align*}
            \abs{f(x)-b}&=\abs[\big]{\sqrt{x}-\sqrt{a}}\\
                        &=\frac{\abs[\big]{\sqrt{x}-\sqrt{a}}\left(\sqrt{x}+\sqrt{a}\right)}{\sqrt{x}+\sqrt{a}}\\
                        &=\frac{\abs{x-a}}{\sqrt{x}+\sqrt{a}}\\
                        &<\frac{\abs{x-a}}{\sqrt{a}}\\
                        &<\frac{\varepsilon\sqrt{a}}{\sqrt{a}}\\
                        &=\varepsilon
        \end{align*}
        Therefore,
        \begin{equation*}
            \lim_{x\to a}\sqrt{x}=\sqrt{a}
        \end{equation*}
    \end{flushleft}
\end{exm}

\begin{rem}\label{rem-undefined-limits}
    Examples of functions where \pref{definition}{def-epsilon-delta-definition-limit} breaks:
    \begin{enumerate}
        \item For any $a\in\mathbb{Z}$, the limit of $f(x)=\floor{x}$ does not exists.
        \item The limit $\displaystyle\lim_{x\to0}\tfrac{1}{x}$ does not exists.
        \item The limit $\displaystyle\lim_{x\to0}\tfrac{1}{x^2}$ does not exists.
        \item The limit $\displaystyle\lim_{x\to0}\sin\left(\tfrac{1}{x}\right)$ does not exists.
    \end{enumerate}
\end{rem}

\begin{thm}\label{thm-limit-arithmetic}
    Let $\displaystyle\lim_{x \to a}f(x) = b_1$ and $\displaystyle\lim_{x \to a}g(x) = b_2$
    be two well-defined limits. Then the following statements hold:
    \begin{enumerate}
        \item $c \cdot f(x) \tolim{x}{a} c \cdot b_1 \quad (c\in\mathbb{R})$
        \item $f(x) + g(x) \tolim{x}{a} b_1 + b_2$
        \item $f(x) \cdot g(c) \tolim{x}{a} b_1 \cdot b_2$
        \item $\tfrac{f(x)}{g(x)} \tolim{x}{a} \tfrac{b_1}{b_2} \quad (b_2 \neq 0)$
    \end{enumerate}
\end{thm}


\begin{proof}
    Of \pref{theorem}{thm-limit-arithmetic}.
    \begin{flushleft}
        \textbf{\nth{2} Property}. From \pref{definition}{def-epsilon-delta-definition-limit}
        we can note that
        \begin{align*}
            &\forall\varepsilon>0\;\exists\delta_1>0:\,\abs{x-a}<\delta_1\implies\abs{f(x)-b_1}<\frac{\varepsilon}{2}\\
            &\forall\varepsilon>0\;\exists\delta_2>0:\,\abs{x-a}<\delta_2\implies\abs{g(x)-b_2}<\frac{\varepsilon}{2}
        \end{align*}        
        Define $\delta\defines\min\{\delta_1,\delta_2\}$ Then, if $\abs{x-a}<\delta$,
        it follows from the triangle inequality that
        \begin{align*}
            \abs{f(x)+g(x)-(b_1+b_2)}&=\abs{(f(x)-b_1)+(g(x)-b_2)}\\
                                     &\leq\abs{f(x)-b_1}+\abs{g(x)-b_2}\\
                                     &<\frac{\varepsilon}{2}+\frac{\varepsilon}{2}\\
                                     &=\varepsilon
        \end{align*}
    \end{flushleft}
\end{proof}

\begin{exm}
    Find the limit of
    \begin{equation*}
        \lim_{x \to 0}\left(\arctan\left(2\sqrt{\frac{\cos(x)}{3x+4}}\right)\right)=\frac{\pi}{4}
    \end{equation*}
    \begin{flushleft}
        \textbf{Answer}: TODO
    \end{flushleft}
\end{exm}

\begin{thm}\label{thm-absolute-value-of-limit:1}
    If $f(x) \tolim{x}{a} 0$, then this is equivalent to
    $\abs{f(x)} \tolim{x}{a} 0$.
\end{thm}

\begin{thm}\label{thm-absolute-value-of-limit:2}
    Let $b\in\mathbb{R}$. If $f(x) \tolim{x}{a} b$, then 
    this implies $\abs{f(x)} \tolim{x}{a} \abs{b}$.
\end{thm}

\begin{rem}
    Note that the opposite direction of \pref{theorem}{thm-absolute-value-of-limit:2}
    is in general not true. A counter example for this would be the 
    \hyperref[def-dirichlet-function]{Dirichlet function}.
\end{rem}

\begin{proof}
    Of \pref{theorem}{thm-absolute-value-of-limit:2}.
    \begin{flushleft}
        Let $\varepsilon>0$. Then $\abs{x-a}<\delta$ implies that
        \begin{align*}
            \abs[\big]{\abs{f(x) - \abs{b}}} &\leq \abs{f(x) - b} \\
                                             &<\varepsilon
        \end{align*}
        by using the reverse triangle inequality for absolute values.
    \end{flushleft}
\end{proof}

\begin{thm}\label{thm-limit-monotonicity}
    Let $f(x) \geq g(x)$ for any $x\in\mathcal{U}_{\varepsilon}^{\bolddot}(a)$. 
    If the limit of $f(x)$ and $g(x)$ exist, then
    \begin{equation*}
        \lim_{x \to a}f(x) \geq \lim_{x \to a}g(x)
    \end{equation*}
\end{thm}

\begin{rem}
    As for \pref{theorem}{thm-limit-monotonicity}, if $f(x) \geq 0$,
    then
    \begin{equation*}
        \lim_{x \to a}f(x) \geq 0
    \end{equation*}
\end{rem}

\begin{rem}
    \pref{Theorem}{thm-limit-monotonicity} does not hold if we replace the greater
    than or equal to inequality with a strictly greater than inequality.
\end{rem}

\begin{thm}\label{thm-unique-limit}
    If the limit of $\displaystyle\lim_{x\to a}f(x)=b\in\mathbb{R}$ exists, then $b$ is unique.
\end{thm}

\begin{proof}
    Of \pref{theorem}{thm-unique-limit}.
    \begin{flushleft}
        Assume \gls{wlog} that $b<c$ are both limits of this function. Define
        $\varepsilon\defines\tfrac{c-b}{3}>0$. Then by 
        \pref{definition}{def-epsilon-delta-definition-limit} we have that
        \begin{equation}\label{eq-unique-limit:1}
            \exists\delta_1\text{ s.t. }a-\delta_1<x<a+\delta_1 \implies b-\varepsilon<f(x)<b+\varepsilon
        \end{equation}
        Likewise we know that
        \begin{equation}\label{eq-unique-limit:2}
            \exists\delta_2\text{ s.t. }a-\delta_1<x<a+\delta_1 \implies c-\varepsilon<f(x)<c+\varepsilon
        \end{equation}
    \end{flushleft}
    Therefore by \pref{equation}{eq-unique-limit:1} and \pref{equation}{eq-unique-limit:2},
    it follows that
    \begin{equation*}
        \left(f(x)<b+\varepsilon<c+\varepsilon\right)\land\left(c+\varepsilon<f(x)\right)
    \end{equation*}
    which is a contradiction. 
\end{proof}

\begin{thm}\label{def-limit-is-bounded}
    If the limit of $\displaystyle\lim_{x\to a}f(x)=b\in\mathbb{R}$ exists, then
    $f$ is bounded in a neighborhood of $a$.
\end{thm}

\begin{thm}\label{thm-sandwich-theorem}
    Suppose that for any $x$ in a neighborhood of $a$
    \begin{equation*}
        h(x) \leq f(x) \leq g(x)
    \end{equation*}
    Further assume that
    \begin{equation*}
        \lim_{x \to a}h(x) = b = \lim_{x \to a}g(x)
    \end{equation*} 
    Then it follows that
    \begin{equation*}
        \lim_{x \to a}f(x) = b
    \end{equation*}
\end{thm}

\begin{proof}
    Of \pref{theorem}{thm-sandwich-theorem}.
    \begin{flushleft}
        Let $\varepsilon>0$. Take $\delta\defines\min\{\delta_1,\delta_2\}$ where
        \begin{align*}
            &0 < \abs{x - a} < \delta_1 \implies \abs{g(x) - b} < \varepsilon \iff b - \varepsilon < g(x) < b + \varepsilon, \\
            &0 < \abs{x - a} < \delta_2 \implies \abs{h(x) - b} < \varepsilon \iff b - \varepsilon < h(x) < b + \varepsilon
        \end{align*}
        Therefore, if $0 < \abs{x - a} < \delta$,
        \begin{equation*}
            b - \varepsilon < h(x) \leq f(x) \leq g(x) < b + \varepsilon
        \end{equation*}
        which is equivalent to
        \begin{equation*}
            \abs{f(x) - b} < \varepsilon \iff \lim_{x \to a}f(x) = b
        \end{equation*}
    \end{flushleft}
\end{proof}

\begin{thm}\label{thm-product-of-bounded-zero-limit}
    If $\displaystyle\lim_{x \to a}f(x) = 0$ and $g(x)$ is bounded in a neighborhood
    of $a$, then
    \begin{equation*}
        \lim_{x \to a} \left(f(x)\cdot g(x)\right) = 0
    \end{equation*}
\end{thm}

\begin{proof}
    Of \pref{theorem}{thm-product-of-bounded-zero-limit}.
    \begin{flushleft}
        Since $g(x)$ is bounded, we can write $\abs{g(x)} \leq M$. So,
        \begin{align*}
            &-M \leq \abs{g(x)} \leq M\\
            \implies
            &-M\cdot\abs{f(x)} \leq \abs{f(x)}\cdot\abs{g(x)} \leq M\cdot \abs{f(x)}\\
            \implies
            &-M\cdot \lim_{x \to a}\abs{f(x)} \leq  \lim_{x \to a}\left(\abs{f(x)}\cdot\abs{g(x)}\right) \leq M \cdot \lim_{x \to a}\abs{f(x)}\\
            \implies
            &-M \cdot 0 \leq \abs{f(x)}\cdot\abs{g(x)} \leq M \cdot 0 && \text{\pref{theorem}{thm-absolute-value-of-limit:1}} \\
            \implies
            & 0 \leq \abs{f(x)}\cdot\abs{g(x)} \leq 0 && \text{\pref{theorem}{thm-limit-arithmetic}}\\
            \implies
            &\lim_{x \to a}\left(\abs{f(x)}\cdot\abs{g(x)}\right)=0 && \text{\pref{theorem}{thm-sandwich-theorem}}\\
            \implies
            &\lim_{x \to a}\left(f(x)\cdot g(x)\right)=0 && \text{\pref{theorem}{thm-absolute-value-of-limit:1}} \\
        \end{align*}
    \end{flushleft}
\end{proof}

\begin{exm}
    The limit of the function 
    \begin{equation*}
        f(x) = x\cdot\sin\left(\frac{1}{x}\right)
    \end{equation*}
    as $x \to 0$ is
    \begin{equation*}
        \lim_{x \to 0}\left(x\cdot\sin\left(\frac{1}{x}\right)\right)=0
    \end{equation*}
    by \pref{theorem}{thm-product-of-bounded-zero-limit} since $\sin\left(\tfrac{1}{x}\right)$
    is bounded\footnote{But in and of itself not defined at $x=0$} by $[-1,1]$ 
    and the left factor of this function is $x=0$ as $x \to 0$.
\end{exm}

\begin{definition}\label{def-one-sided-limits}
    We denote the one-sided limit of the function $f(x)$ approached from the right by
    \begin{equation*}
        \lim_{x \to a^+}f(x)=b
    \end{equation*}
    if and only if
    \begin{equation}
        \forall\varepsilon>0\;\exists\delta>0:x\in(a,a+\delta)\implies\abs{f(x)-b}<\varepsilon
    \end{equation}
    Conversely, the left-sided limit is denoted by
    \begin{equation*}
        \lim_{x \to a^-}f(x)=b
    \end{equation*}
    if and only if
    \begin{equation}
        \forall\varepsilon>0\;\exists\delta>0:x\in(a-\delta,a)\implies\abs{f(x)-b}<\varepsilon
    \end{equation}
\end{definition}

\begin{rem}\label{rem-one-sided-limits}
    With respect to \pref{definition}{def-one-sided-limits}, note that there exist
    several equivalent notations. So, the right-sided limit of $f(x)$ can be denoted
    by
    \begin{equation*}
        \lim_{x \to a^+}f(x)=b
        \iff
        \lim_{x \underset{x>a}{\to} a}f(x)=b
        \iff
        \lim_{x\searrow  a}f(x)=b
    \end{equation*}
    Similarly, at times the left-sided limits may be denoted by
    \begin{equation*}
        \lim_{x \to a^-}f(x)=b
        \iff
        \lim_{x \underset{x<a}{\to} a}f(x)=b
        \iff
        \lim_{x \nearrow a}f(x)=b
    \end{equation*}
\end{rem}

\begin{thm}\label{thm-limit-exists-one-sided-limits}
    The limit of $f(x) \to b$ exists as $x \to a$ if and only if the left-sided
    limit as well as the right-sided limit of $f$ exists with
    \begin{equation}
        \lim_{x \to a}f(x)=b \iff \lim_{x \to a^+}f(x)=b=\lim_{x \to a^-}f(x)
    \end{equation}
\end{thm}

\begin{proof}
    Of \pref{theorem}{thm-limit-exists-one-sided-limits}.
    \begin{flushleft}
        TODO
    \end{flushleft}
\end{proof}

\begin{exm}\label{exm-important-sin-over-x-limit}
    Show that\footnote{This is a very useful limit that we are going to encounter
    more often in the near future.}
    \begin{equation}\label{eq-important-sin-over-x-limit}
        \lim_{x \to 0}\frac{\sin(x)}{x}=1
    \end{equation}
    \begin{flushleft}
        \textbf{Answer}: From figure (??) we can derive two observations:
        \begin{equation*}
            \forall x\in\left(0,\frac{\pi}{2}\right)\implies \sin(x)<x
        \end{equation*}
        Notice also that
        \begin{equation*}
            x < \tan(x)
        \end{equation*}
        So, this implies
        \begin{equation}\label{eq-important-sin-over-x-limit:1}
            \sin(x) < x < \tan(x)
        \end{equation}
        Furthermore, figure (??) reveals that
        \begin{equation*}
            \forall x\in\mathbb{R}:\abs{\sin(x)}\leq\abs{x}
        \end{equation*}
        From \pref{equation}{eq-important-sin-over-x-limit:1} follows that for all
        $\tfrac{\pi}{2}>x>0$:
        \begin{align*}
            \implies 
            &\frac{1}{\sin(x)} > \frac{1}{x} > \frac{1}{\tan(x)} \\
            \implies 
            &1 > \frac{\sin(x)}{x} > \cos(x) \\
            \implies 
            &1 > \lim_{x \to 0^+}\frac{\sin(x)}{x} > \lim_{x \to 0^+}\cos(x) \\
            \implies
            &1 > \lim_{x \to 0^+}\frac{\sin(x)}{x} > 1\\
            \implies 
            &\lim_{x \to 0^+}\frac{\sin(x)}{x}=1 && \text{theorem (\ref{thm-sandwich-theorem})}
        \end{align*}
        A similar argument can be made for 
        \begin{align*}
            \forall x\in\left(-\frac{\pi}{2},0\right):\lim_{x \to 0^-}\frac{\sin(x)}{x}=1
        \end{align*}
        Last but not least, \pref{theorem}{thm-limit-exists-one-sided-limits} ensures
        that the limit in \pref{equation}{eq-important-sin-over-x-limit} exists.
    \end{flushleft}
\end{exm}

\begin{thm}\label{thm-monotone-one-sided-limits}
    Assume that $f$ is monotone on some interval $[a,b]$. Then $f$ has one-sided
    limits at every point.
\end{thm}

\begin{definition}\label{def-infinity-limits}
    Let $f$ be a function such that\footnote{In other words, $\infty$ is a limit
    point of the domain of $f$}
    \begin{equation}
        \forall c\in\mathbb{R}: (c,\infty)\cap\domain{f}\neq\emptyset
    \end{equation}
    Next let $b\in\mathbb{R}$. We say that $f(x) \to b$ for $x \to \infty$ if and only if
    \footnote{Similarly, we can define the limit for $f(x) \to b$ as $x \to -\infty$}
    \begin{equation}
        \bigwedge_{\varepsilon>0}\bigvee_{c\in\mathbb{R}}\bigwedge_{x\in\domain{f}}
        \left(x>c \implies \abs{f(x) - b}<\varepsilon\right)
    \end{equation}
    Then this is equivalent to \cite[p.70]{wuest2009}
    \begin{equation}
        \lim_{x \to \infty}f(x)=b
    \end{equation}
\end{definition}

\begin{rem}
    Definition (\ref{def-infinity-limits}) works with all previously encountered theorems.
\end{rem}

\begin{exm}\label{exm-infinity-limit:1}
    Let $f$ be defined by \cite[p.71]{wuest2009}
    \begin{equation*}
        f(x)\defines\frac{3x^2-2x+1}{x^2+5x}
    \end{equation*}
    where $x\in(0,\infty)$. Show that
    \begin{equation*}
        \lim_{x \to \infty}f(x)=3
    \end{equation*} 
    \begin{flushleft}
        \textbf{Answer}: First observe, that for very large $x$ the quadratic terms
        in the numerator and denominator dominate over the other terms in the long
        run. Keeping this in mind we can make an intelligent guess for the limit point
        $f(x) \to 3$ as $x \to \infty$. Now for the formal part of this proof: Let $x>0$.
        Then, for $a(\varepsilon)\defines\max\left\{1,\tfrac{18}{\varepsilon}\right\}$ it
        follows that
        \begin{align*}
            \abs{f(x) - b} &= \abs[\Bigg]{\frac{3x^2-2x+1}{x^2+5x}-3}\\
                           &= \abs[\Bigg]{\frac{3x^2-2x+1-3x^2-15x}{x^2+5x}}\\
                           &= \abs[\Bigg]{\frac{-17x+1}{x^2+5x}}\\
                           &\leq {\frac{\abs{-17x}+\abs{1}}{x^2+5x}}\\
                           &= \frac{17x}{x^2+5x} + \frac{1}{x^2+5x}\\
                           &\leq \frac{17x}{x^2} + \frac{1}{x^2}\\
                           &\leq \frac{18}{x} && \text{if } x\geq1\\
                           &< \frac{18}{\tfrac{18}{\varepsilon}}\\
                           &= \varepsilon
        \end{align*}
        in other words for every $\varepsilon>0$ there exists a $c\in\mathbb{R}$
        such that $c\defines a(\varepsilon)$ for which is true that
        \begin{equation*}
            x>c \implies \abs{f(x)-b}<\varepsilon
        \end{equation*}
        which means that $f(x)$ converges towards $3$ as $x$ approaches infinity.
    \end{flushleft}
\end{exm}

\begin{definition}\label{def-infinite-limits}
    We say a function has an infinite limit at infinity when
    \begin{equation}
        \bigwedge_{c\in\mathbb{R}}\bigvee_{\delta>0}
        \left(\abs{x-a}<\delta \implies f(x)>c\right)
    \end{equation}
    which is denoted by
    \begin{equation}
        \lim_{x \to a}f(x)=\infty
    \end{equation}
    Another frequently used expression to describes this behavior is divergence;
    limits that exhibit this type behavior are said to diverge. 
\end{definition}

\begin{exm}\label{exm-infinity-limit:3}
    Use \pref{definition}{def-infinite-limits} to show that
    \begin{equation}
        \lim_{x \to 0}\frac{1}{x^2}=\infty
    \end{equation}
    \begin{flushleft}
        \textbf{Answer}: Let $c\in\mathbb{R}^+$ such that $\delta\defines\tfrac{1}{\sqrt{c}}$.
        Then, it follows that
        \begin{equation*}
            0<\abs{x-0}<\delta \implies \abs{x} < \frac{1}{\sqrt{c}} \implies \frac{1}{x^2} > c
        \end{equation*}
    \end{flushleft}
\end{exm}

\begin{rem}
    Beware: for $f(x) \to \pm\infty$ not all theorems hold, in particular 
    \pref{theorem}{thm-limit-arithmetic} is only partially true.
\end{rem}

\begin{thm}\label{thm-pizza-theorem}
    If $g(x) \geq f(x)$ in a neighborhood of $a$ and $f(x) \to \infty$ as $x \to a$,
    then $g(x) \to \infty$ for $x \to a$.
\end{thm}
