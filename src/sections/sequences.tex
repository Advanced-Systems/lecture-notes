\subsection{Sequences}\label{subsec-sequences}

\begin{definition}\label{def-sequence}
	An infinite ordered set of real\footnote{This document restricts the definition
		to the set of real numbers, but in other courses it can be more abstract than that}
	numbers is called a sequence, most formally denoted by $\{a_n\}_{n=1}^\infty$,
	or just $a_n$.
\end{definition}

\begin{definition}\label{def-sequence-recursive:1}
	Let $a\in\mathbb{N}$. We define \cite[p.51]{wuest2009}
	\begin{align*}
		a_1     & = a + 1   \\
		a_{n+1} & = a_n + 1
	\end{align*}
\end{definition}

\begin{definition}\label{def-sequence-recursive:2}
	Let $a\in\mathbb{R}\setminus\{0\}$. We define \cite[p.51]{wuest2009}
	\begin{align*}
		a^0     & = 1                                                                         \\
		a^{n+1} & = a^n \cdot n &  & (n\in\mathbb{N}_0\defines\mathbb{N}\cup\{0\}=\mathbb{Z})
	\end{align*}
\end{definition}

\begin{definition}\label{def-sequence-recursive:3}
	Let $n\in\mathbb{N}_0$. We define \cite[p.51]{wuest2009}
	\begin{align*}
		0!     & = 1       \\
		(n+1)! & = n!(n+1)
	\end{align*}
	This is known as the faculty.
\end{definition}

\begin{definition}\label{def-sequence-recursive:4}
	Let $c\in\mathbb{Z}$ and $\{a_n\}_{n\in\mathbb{Z}_c}$ be sequences. We define \cite[p.51]{wuest2009}
	\begin{align*}
		\sum_{k=c}^n a_k     & = a_c                                                \\
		\sum_{k=c}^{n+1} a_k & = \sum_{k=c}^n a_k + a_{n+1} &  & (n\in\mathbb{Z}_c)
	\end{align*}
	This is known as the sum.
\end{definition}

\begin{definition}\label{def-sequence-recursive:5}
	Let $c\in\mathbb{Z}$ and $\{a_n\}_{n\in\mathbb{Z}_c}$ be sequences. We define \cite[p.51]{wuest2009}
	\begin{align*}
		\prod_{k=c}^n a_k     & = a_c                                                                  \\
		\prod_{k=c}^{n+1} a_k & = \left(\prod_{k=c}^n a_k\right) \cdot a_{n+1} &  & (n\in\mathbb{Z}_c)
	\end{align*}
	This is known as the product.
\end{definition}

\begin{rem}
	A sequence can also be written as a function $f:\mathbb{N}\to\mathbb{R}$,
	where $a_n = f(n)$.
\end{rem}

\begin{definition}\label{def-sequence-limit}
	Let $a_n$ be a sequence. We say that
	\begin{equation*}
		\lim_{n\to\infty}a_n=b
	\end{equation*}
	for $b\in\mathbb{R}$ if and only if
	\begin{equation*}
		\forall\varepsilon>0\;\exists N\in\mathbb{N}: n>N \implies \abs{a_n - b}<\varepsilon
	\end{equation*}
\end{definition}

\begin{rem}\label{rem-remarkable-sequences}
	What follows is a short list of some of the more notable sequences that we will
	get to know over time:
	\begin{enumerate}
		\item Linear sequence: $a_n=n$, \textit{e.g.} $1,3,4,5,\dots$
		\item Harmonic sequence: $a_n=\tfrac{1}{n}$, \textit{e.g.} $1,\tfrac{1}{2},\tfrac{1}{3},\tfrac{1}{4},\tfrac{1}{5},\dots$
		\item Alternating sequence: $a_n=(-1)^n$, \textit{e.g.} $-1,1,-1,1,-1,\dots$
		\item Alternating harmonic sequence: $a_n=\tfrac{(1)^{n}}{n}$, \textit{e.g.} $1,-\tfrac{1}{2},\tfrac{1}{3},-\tfrac{1}{4},\tfrac{1}{5},\dots$
		\item Arithmetic sequence: $a_n=5+2(n-1)$, \textit{e.g.} $5,7,9,11,13,\dots$
		\item Geometric sequence: $a_n=3\cdot2^{n-1}$, \textit{e.g.} $3,6,12,24,48,\dots$
		\item Constant sequence: $a_n=2$, \textit{e.g.} $2,2,2,2,2,\dots$
		\item Case sequence: $a_n=\begin{cases}1\text{ if }n\text{ is even}\\n\text{ else}\end{cases}$, \textit{e.g.} $1,1,3,1,5,\dots$
		\item Sequence of prime numbers\footnote{To date there exists no known formula for this sequence}, \textit{e.g.} $2,3,5,7,11,\dots$
		\item Sequence of digital digits of $\pi$, \textit{e.g.} $1,4,1,5,9,\dots$
		\item Recursive sequence: $a_n=\begin{cases}a_1=2\\a_n=3a_{n-1}^2\end{cases}$, \textit{e.g.} $2,12,432,559.872,\dots$
	\end{enumerate}
\end{rem}

\begin{exm}\label{exm-harmonic-sequence}
	Let $a_n$ be the harmonic sequence. Show that $\tfrac{1}{n}\to0$ as $n\to\infty$.
	\begin{flushleft}
		\textbf{Answer}: Let $\varepsilon>0$ and $N>\tfrac{1}{\varepsilon}$. Then,
		\begin{align*}
			\abs{f(x) - b} & = \abs[\Bigg]{\frac{1}{n}-0} \\
			               & = \frac{1}{n}                \\
			               & < \frac{1}{N}                \\
			               & < \varepsilon
		\end{align*}
	\end{flushleft}
\end{exm}

\begin{rem}
	Adding, dropping or changing any finite number of elements in a sequence does
	not affect the limit.
\end{rem}

\begin{thm}\label{thm-sequence-unique-limit}
	If $a_n$ has a limit, then the limit is unique.
\end{thm}

\begin{thm}\label{thm-sequence-limit-bounded}
	If $a_n$ has a limit, then $a_n$ is also bounded.
\end{thm}

\begin{rem}\label{rem-sequence-limit-bounded}
	\pref{Theorem}{thm-sequence-limit-bounded} only works in one direction; the
	limit might not exist if the sequence is bounded. Take for instance the alternating
	sequence as an counter example. Conversely, if the sequence is not bounded, then
	it has no finite limit.
\end{rem}

\begin{proof}
	Of \pref{theorem}{thm-sequence-limit-bounded}.
	\begin{flushleft}
		Suppose that $a_n$ has a limit, \textit{i.e.}
		\begin{equation*}
			\lim_{n\to\infty}a_n=b
		\end{equation*}
		then define
		\begin{equation*}
			M\defines\max\left\{b+1,a_1,a_2,\dots,a_N\right\}
		\end{equation*}
		for $n>N$ such that $a_n\in(b-1,b+1)$. We define
		\begin{equation*}
			m\defines\min\left\{b-1,a_1,a_2,\dots,a_N\right\}
		\end{equation*}
		Hence, $m \leq a_n \leq M$ for all $n\in\mathbb{N}$.
	\end{flushleft}
\end{proof}

\begin{thm}\label{thm-sequence-arithmetic}
	Let $a_n \seqinfty{n} L$ and $b_n \seqinfty{n} K$ be two limited sequences. Then
	\begin{enumerate}
		\item $\forall c\in\mathbb{R}:c \cdot a_n \seqinfty{n} c \cdot L$
		\item $a_n+b_n \seqinfty{n} L+K$
		\item $a_n \cdot b_n \seqinfty{n} L \cdot K$
		\item $\tfrac{a_n}{b_n} \seqinfty{n} \tfrac{L}{K}$ if $b_n \neq 0$ and $K \neq 0$
	\end{enumerate}
\end{thm}

\begin{exm}\label{exm-sequence-arithmetic:1}
	By \pref{example}{exm-harmonic-sequence} and \pref{theorem}{thm-sequence-arithmetic}
	it immediately follows that
	\begin{equation*}
		1+\frac{1}{n} \seqinfty{n} 1
	\end{equation*}
\end{exm}

\begin{thm}\label{thm-sequence-sqrt-limit}
	Let $a_n \seqinfty{n} b$. Assume that for any $a_n\geq0$ ($n\in\mathbb{N}$),
	\begin{equation*}
		\sqrt{a_n} \seqinfty{n} \sqrt{b}
	\end{equation*}
\end{thm}

\begin{thm}\label{thm-sequence-abs}
	Let $a_n \seqinfty{n} b$. Then
	\begin{equation*}
		\abs{a_n} \seqinfty{n} \abs{b}
	\end{equation*}
\end{thm}

\begin{thm}\label{thm-sequence-zero-limit}
	Let $a_n \seqinfty{n} 0$. Then this is equivalent to
	\begin{equation*}
		\abs{a_n} \seqinfty{n} 0
	\end{equation*}
\end{thm}

\begin{rem}\label{rem-sequence-abs}
	Note that for \pref{theorem}{thm-sequence-abs},
	\begin{equation*}
		\abs{a_n} \seqinfty{n} \abs{b} \notimplies a_n \seqinfty{n} b
	\end{equation*}
	The alternating sequence from \pref{remark}{rem-remarkable-sequences} is a
	counterexample for this statement.
\end{rem}

\begin{exm}\label{exm-sequence-arithmetic:2}
	Find the limit of
	\begin{equation*}
		\lim_{n\to\infty}\sqrt{\frac{n+1}{3n-\frac{2}{n}}}
	\end{equation*}
	\begin{flushleft}
		\textbf{Answer}: We use the result from \pref{example}{exm-harmonic-sequence},
		\textit{i.e.} $\tfrac{1}{n} \seqinfty{n} 0$ and some basic algebraic manipulation to get
		\begin{align*}
			\lim_{n\to\infty}\sqrt{\frac{n+1}{3n-\frac{2}{n}}}
			 & = \lim_{n\to\infty}\sqrt{\frac{1+\frac{1}{n}}{3-\frac{2}{n^2}}}                                                                                                                    \\
			 & = \lim_{n\to\infty}\sqrt{\frac{1+\frac{1}{n}}{3-2\cdot\frac{1}{n}\cdot\frac{1}{n}}}                                                                                                \\
			 & = \frac{1}{\sqrt{3}}                                                                &  & \text{\pref{theorem}{thm-sequence-arithmetic} \& \pref{theorem}{thm-sequence-zero-limit}}
		\end{align*}
	\end{flushleft}
\end{exm}

\begin{thm}\label{thm-product-of-bounded-zero-sequence}
	If $a_n \seqinfty{n} 0$ and $b_n$ is a bounded sequence, then\footnote{This is
		analogous to \pref{theorem}{thm-product-of-bounded-zero-limit}}
	\begin{equation*}
		a_n \cdot b_n \seqinfty{n} 0
	\end{equation*}
\end{thm}

\begin{thm}\label{thm-sequence-sandwich-theorem}
	Assume that
	\begin{equation*}
		a_n \leq b_n \leq c_n
	\end{equation*}
	as well as
	\begin{equation*}
		\lim_{n\to\infty} a_n = L = \lim_{n\to\infty} c_n
	\end{equation*}
	Then the sandwich theorem for sequences states that
	\begin{equation*}
		\lim_{n\to\infty} b_n = L
	\end{equation*}
\end{thm}

\begin{exm}
	Take the sequence $a_n=\tfrac{1}{n}\sin(n)$. We can find lower and upper bounds
	for this sequence since $\codomain{\sin(n)}=[-1,1]$, such that
	\begin{equation*}
		-\frac{1}{n}\leq\frac{\sin(n)}{n}\leq\frac{1}{n}
	\end{equation*}
	Notice that by \pref{theorem}{thm-sequence-zero-limit},
	\begin{equation*}
		\abs[\Bigg]{-\frac{1}{n}} \seqinfty{n} 0
	\end{equation*}
	Hence, by \pref{theorem}{thm-sequence-sandwich-theorem}
	\begin{equation*}
		\lim_{n\to\infty}\frac{\sin(n)}{n}=0
	\end{equation*}
\end{exm}

\begin{exm}
	Let $a_n=\tfrac{n!}{n^n}$. We claim that $a_n$ is bounded from above by the
	sequence $\tfrac{1}{n}$ since
	\begin{align*}
		\frac{n!}{n^n}                                                                           & \leq\frac{1}{n}                                         \\
		\iff
		n!n                                                                                      & \leq n^n                                                \\
		\iff
		n!                                                                                       & \leq n^{n-1}                                            \\
		\iff
		\underbrace{n \cdot (n-1) \cdot (n-2) \cdots 3 \cdot 2 \cdot 2 \cdot 1}_{n\text{ times}} & \leq \underbrace{n \cdot n \cdots n}_{n-1\text{ times}} \\
		\iff
		\underbrace{n \cdot (n-1) \cdot (n-2) \cdots 3 \cdot 2 \cdot 2}_{n-1\text{ times}}       & \leq \underbrace{n \cdot n \cdots n}_{n-1\text{ times}} \\
	\end{align*}
	The sequence is also bounded from below by $0$, so by theorem (\ref{thm-sequence-sandwich-theorem})
	it follows that
	\begin{equation*}
		0\leq\frac{n!}{n^n}\leq\frac{1}{n}\implies\lim_{n\to\infty}\frac{n!}{n^n}=0
	\end{equation*}
\end{exm}

\begin{thm}\label{thm-sequence-nth-root}
	Let $a_n=\sqrt[n]{n}$. Then this sequence converges to
	\begin{equation*}
		\sqrt[n]{n} \seqinfty{n} 1
	\end{equation*}
\end{thm}

\begin{thm}\label{thm-sequence-nth-root-of-constant}
	Let $a_n=\sqrt[n]{c}$ for any $c\in\mathbb{R}^+\setminus\{0\}$. Then this
	sequence converges to
	\begin{equation*}
		\sqrt[n]{c} \seqinfty{n} 1
	\end{equation*}
\end{thm}

\begin{exm}\label{exm-sequence-nth-root}
	Consider the sequence
	\begin{equation*}
		a_n =\sqrt[n]{3^{2n+1} \cdot n}
	\end{equation*}
	Then this is the same as
	\begin{equation*}
		a_n = \sqrt[n]{9^n}\sqrt[n]{3}\sqrt[n]{n} = 9\sqrt[n]{3}\sqrt[n]{n}
	\end{equation*}
	So, by \pref{theorem}{thm-sequence-nth-root} and \pref{theorem}{thm-sequence-nth-root-of-constant}
	this implies
	\begin{equation*}
		a_n \seqinfty{n} 9
	\end{equation*}
\end{exm}

\begin{thm}\label{thm-sequence-converges-positively}
	Let $a_n\geq0$. Then\footnote{This is still true if we only require $a_n\geq0$
		for some threshold $n\geq N\in\mathbb{N}$.},
	\begin{equation*}
		a_n \seqinfty{n} b \implies b \geq 0
	\end{equation*}
\end{thm}

\begin{thm}\label{thm-sequence-limit-greater-than}
	Suppose that $a_n \geq b_n$. Let $a_n \seqinfty{n} K$ and $b_n \seqinfty{n} L$. Then
	\footnote{This would be no longer true for strict inequalities, \textit{cf.} the
		sequence $0<\tfrac{1}{n}\seqinfty{n}0$.},
	\begin{equation*}
		K = \lim_{n\to\infty} a_n \geq \lim_{n\to\infty} b_n = L
	\end{equation*}
\end{thm}

\begin{thm}\label{thm-sequence-greater-than}
	If $a_n \seqinfty{n} L$ and $b_n \seqinfty{n} K$ with $L > K$, then there exists
	a threshold $n>N\in\mathbb{N}$ such that $a_n>b_n$.
\end{thm}

\begin{definition}\label{def-sequence-divergence}
	We say\footnote{Also called divergence of a sequence} that $a_n \seqinfty{n} \infty$
	if for every $M$ there exists an $N\in\mathbb{N}$ such that
	\begin{equation*}
		n>N \implies a_n > M
	\end{equation*}
\end{definition}

\begin{rem}
	Here are some sequences that diverge towards infinity:
	\begin{enumerate}
		\item $a_n=n!$
		\item $a_n=2^n$
		\item $a_n=n^3$
	\end{enumerate}
	Another very interesting sequence is $a_n=(-1)^n\cdot n$. It's an unbounded
	alternating sequence that bounces between $-\infty$ for odd numbers, and
	$+\infty$ for even numbers.
\end{rem}

\begin{thm}\label{thm-sequence-infinity-zero}
	If $a_n \seqinfty{n} \infty$, then $\tfrac{1}{a_n} \seqinfty{n} 0$.
\end{thm}

\begin{thm}\label{thm-sequence-zero-infinity}
	If $a_n>0$ for all $n\in\mathbb{N}$ where $a_n \seqinfty{n} 0$, then
	$\tfrac{1}{a_n} \seqinfty{n} \infty$.
\end{thm}

\begin{thm}
	If $a_n \geq b_n$ and $b_n \seqinfty{n} \infty$, then $a_n \seqinfty{n} \infty$.
\end{thm}

\begin{definition}\label{def-monotonicity-sequences}
	Let $a_n$ be a sequence. It is called
	\begin{enumerate}
		\item monotonically increasing, if $\forall n>N\in\mathbb{N}:a_{n+1} \geq a_n$
		\item strictly monotonically increasing, if $\forall n>N\in\mathbb{N}:a_{n+1} > a_n$
		\item monotonically decreasing, if $\forall n>N\in\mathbb{N}:a_{n+1} \leq a_n$
		\item strictly monotonically decreasing, if $\forall n>N\in\mathbb{N}:a_{n+1} < a_n$
	\end{enumerate}
\end{definition}

\begin{exm}\label{exm-monotonicity-sequences:1}
	We claim that $a_n=\tfrac{n^2}{2^n}$ is a monotonically decreasing sequence for $n>3$.
	\begin{flushleft}
		\textbf{Answer}: To prove this, we need to show that $a_{n+1}<a_n$. So,
		\begin{align*}
			\frac{a_{n+1}}{a_n} & = \frac{(n+1)^2}{2^{n+1}}\cdot\frac{2^n}{n^2}           \\
			                    & = \frac{1}{2}\left(\frac{n+1}{n}\right)^2               \\
			                    & = \frac{1}{2}\left(1+\frac{1}{n}\right)^2               \\
			                    & < 1                                           & (\star)
		\end{align*}
		The inequality in $(\star)$ holds \textit{iff} $n>3:\left(1+\tfrac{1}{n}\right)^2<2$.
		But then
		\begin{align*}
			\frac{1}{n}                  & \leq \frac{1}{3}                  \\
			\implies
			1+\frac{1}{n}                & \leq 1+\frac{1}{3}                \\
			\implies
			\left(1+\frac{1}{n}\right)^2 & \leq \left(1+\frac{1}{3}\right)^2 \\
			\implies
			\left(1+\frac{1}{n}\right)^2 & \leq \frac{16}{9} = 2
		\end{align*}
	\end{flushleft}
\end{exm}

\begin{exm}\label{exm-monotonicity-sequences:2}
	We claim that $a_n=\tfrac{n!}{n^n}$ is a strictly monotonically decreasing sequence.
	\begin{flushleft}
		\textbf{Answer}: To prove this, we need to show that $a_{n+1}<a_n$. So,
		\begin{align*}
			\frac{a_{n+1}}{a_n} & = \frac{(n+1)!}{(n+1)^{n+1}}\cdot\frac{n^n}{n!} \\
			                    & = \frac{(n+1)n^n}{(n+1)^{n+1}}                  \\
			                    & = \left(\frac{n}{n+1}\right)^n                  \\
			                    & < 1
		\end{align*}
		wherefore $a_{n+1}<a_n$ for any $n\in\mathbb{N}$.
	\end{flushleft}
\end{exm}

\begin{thm}\label{thm-euler-sequence-monotonicity-increasing}
	The sequence $a_n=\left(1+\tfrac{1}{n}\right)^n$ is strictly monotonically increasing.
\end{thm}

\begin{thm}\label{thm-monotone-bounded-sequence-converges}
	Every monotone and bounded sequence converges.
\end{thm}

\begin{rem}\label{rem-euler-sequence}
	By \pref{theorem}{thm-monotone-bounded-sequence-converges}, the sequence in
	\pref{theorem}{thm-euler-sequence-monotonicity-increasing} has a limit bounded
	by $2 \leq a_n \leq 3$ for all $n\in\mathbb{N}$ and is denoted by $e$, which
	is an irrational number.
\end{rem}

\begin{proof}
	Of \pref{theorem}{thm-monotone-bounded-sequence-converges}.
	\begin{flushleft}
		Assume \gls{wlog} that the sequence $a_n$ is monotonically increasing.
		By the axiom of completeness, $a_n$ has a supremum by the virtue of being
		bounded, denoted by $L=\sup(a_n)$. We claim that
		\begin{equation*}
			\lim_{n\to\infty}a_n=L
		\end{equation*}
		Let $\varepsilon>0$. Note that $L-\varepsilon$ is not an upper bound of
		$a_n$. Therefore, for any $n>N\in\mathbb{N}$ we have that
		\begin{equation*}
			\exists a_N: L - \varepsilon < a_N < a_n < \sup(a_n) = L < L + \varepsilon
		\end{equation*}
		By definition (\ref{def-sequence-limit}),
		\begin{equation*}
			L - \varepsilon < a_n < L + \varepsilon \iff \abs{a_n - L} < \varepsilon
		\end{equation*}
	\end{flushleft}
\end{proof}

\begin{thm}\label{thm-monotone-sequence-converges-diverges}
	Every monotone sequence either converges or diverges towards $-\infty$ or $+\infty$.
\end{thm}

\begin{rem}
	Based on \pref{theorem}{thm-euler-sequence-monotonicity-increasing} here are
	a few notable observations:
	\begin{equation}
		\lim_{n\to\infty}\left(1+\frac{a}{n}\right)^n=e^a
	\end{equation}
	Additionally, if $a_n\neq0$ for every $n\in\mathbb{N}$ and $a_n \seqinfty{n} \infty$, then
	\begin{equation}
		\lim_{n\to\infty}\left(1+\frac{1}{a_n}\right)^{a_n}=e
	\end{equation}
\end{rem}

\begin{exm}\label{exm-sequence-limit:1}
	Consider the sequence
	\begin{equation*}
		a_n=\sqrt[n+1]{\left(\frac{n}{n-1}\right)^{n^2-n}}
	\end{equation*}
	Find the limit of $a_n$.
	\begin{flushleft}
		\textbf{Answer}:
		\begin{align*}
			\lim_{n\to\infty}\sqrt[n+1]{\left(\frac{n}{n-1}\right)^{n^2-n}}
			 & =\lim_{n\to\infty}\left(\frac{n}{n-1}\right)^{\frac{n^2-n}{n+1}}                      \\
			 & =\lim_{n\to\infty}\left(\frac{n-1+1}{n-1}\right)^{\frac{n(n-1)}{n+1}}                 \\
			 & =\lim_{n\to\infty}\left(\left(1+\frac{1}{n-1}\right)^{n-1}\right)^{\frac{n+1-1}{n+1}} \\
			 & =\lim_{n\to\infty}\left(\left(1+\frac{1}{n-1}\right)^{n-1}\right)^{1-\frac{1}{n+1}}   \\
			 & =e
		\end{align*}
	\end{flushleft}
\end{exm}

\begin{exm}\label{exm-sequence-limit:2}
	Consider the sequence
	\begin{equation*}
		a_n=\begin{cases}
			a_1=\frac{1}{4} \\
			a_n=(a_{n-1})^2+\frac{1}{4}
		\end{cases}
	\end{equation*}
	Find the limit of $a_n$.
	\begin{flushleft}
		\textbf{Answer}:
		First we need to show by induction, that the sequence is monotonically increasing
		and bounded from above by $\tfrac{1}{2}$.
		\begin{flushleft}
			\textbf{Monotonically increasing}: The base case holds since
			\begin{equation*}
				a_2 = \left(\frac{1}{4}\right)^2+\frac{1}{4} = \frac{1}{2} > \frac{1}{4} = a_1
			\end{equation*}
			For the induction hypothesis, assume that $a_{n+1}>a_{n}$. Then this equivalent
			to $a_{n+1}-a_n>0$. Therefore, the induction step $n \to n+1$ it follows that
			\begin{align*}
				a_{n+2}-a_{n+1} & = \left(a_{n+1}^2+\frac{1}{4}\right) - \left(a_n^2+\frac{1}{4}\right)                                  \\
				                & = a_{n+1}^2 - a_n^2                                                                                    \\
				                & = (a_{n+1} - a_n)(a_{n+1} + a_n)                                      &  & \text{Induction Hypothesis} \\
				                & >0
			\end{align*}
			Since the assertion works for the base case, and assuming it works for the
			induction hypothesis as well, the sequence is monotonically increasing for all $n\in\mathbb{N}$
			by the principle of induction.
		\end{flushleft}
		\begin{flushleft}
			\textbf{Bounded from above}: The base case holds since
			\begin{equation*}
				a_1 = \frac{1}{4} < \frac{1}{2}
			\end{equation*}
			For the induction hypothesis, assume that $a_{n}<\tfrac{1}{2}$. Then
			the induction step $n \to n+1$ indicates that
			\begin{align*}
				a_{n+1} & = (a_n)^2 + \frac{1}{4}                                                     \\
				        & < \left(\frac{1}{2}\right)^2 + \frac{1}{4} &  & \text{Induction Hypothesis} \\
				        & = \frac{1}{2}
			\end{align*}
			Since the assertion works for the base case, and assuming it works for the
			induction hypothesis as well, the sequence is bounded from above by $\tfrac{1}{2}$
			for all $n\in\mathbb{N}$ by the principle of induction.
		\end{flushleft}
		So, by \pref{theorem}{thm-monotone-bounded-sequence-converges}, the sequence
		converges. Therefore, $a_n \seqinfty{n} L$ and $a_{n-1} \seqinfty{n} L^2+\tfrac{1}{4}$, \textit{i.e.}
		\begin{align*}
			 & L = L^2 + \frac{1}{4}   \\
			\implies
			 & L - L^2 - \frac{1}{4}=0 \\
			\implies
			 & L=\frac{1}{2}
		\end{align*}
	\end{flushleft}
\end{exm}
