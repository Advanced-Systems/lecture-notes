\subsection{Linear Combinations}\label{subsec-linear-combinations}

\begin{definition}\label{def-linear-combinations}
	Let $\mathcal{V}$ be a vector space with elements $v_1, v_2, \dots, v_n$
	with $n\in\mathbb{N}$. A vector $v\in\mathcal{V}$ with coefficients
	$\lambda_i\in\mathcal{F}$ of the form
	\begin{equation}
		v=\lambda_1 v_1 + \dots + \lambda_n v_n = \sum_{i=1}^n \lambda_i v_i \in\mathcal{V}
	\end{equation}
	is called a \textbf{linear combination} of $v_1, v_2, \dots, v_n$.
\end{definition}

\begin{exm}
	Let $v_1=\inlinematrix{1\\2\\3}$, $v_2=\inlinematrix{2\\-1\\-2}$,
	$v_3=\inlinematrix{3\\1\\1}$, and $v_4=\inlinematrix{4\\0\\2}$. Then
	$v=\inlinematrix{13\\8\\13}$ is a \textit{linear combination} of
	$v_1,v_2,v_3,v_4$ since
	\begin{equation*}
		\begin{pmatrix}
			13 \\8\\13
		\end{pmatrix}=
		0\begin{pmatrix}
			1 \\2\\3
		\end{pmatrix}-
		3\begin{pmatrix}
			2 \\-1\\-2
		\end{pmatrix}+
		5\begin{pmatrix}
			3 \\1\\1
		\end{pmatrix}+
		1\begin{pmatrix}
			4 \\0\\2
		\end{pmatrix}
	\end{equation*}
\end{exm}

\begin{definition}\label{def-linear-span}
	Expanding on the notion of a linear combination which was introduced in
	\pref{definition}{def-linear-combinations}, the linear span is defined as
	the set
	\begin{equation}
		\fspan\{v_1, \dots, v_n\}\defines
		\left\{
		\lambda_1,\dots,\lambda_n\in\mathcal{F}\setbuild
		\sum_{i=1}^n \lambda_i v_i
		\right\}
	\end{equation}
	i.e. the collection of all linear combinations.
\end{definition}

\begin{thm}\label{thm-span-is-subspace}
	$\fspan\{v_1, \dots, v_n\}$ is a subspace of $\mathcal{V}$.
\end{thm}

\begin{proof}
	Of \pref{theorem}{thm-span-is-subspace}:
	Let  $\mathcal{W}=\fspan\{v_1, \dots, v_n\}$ and $\lambda\in\mathcal{F}$. Then,
	\begin{enumerate}
		\item[(i)] $\mathcal{W}\neq\emptyset$ since all the $v_i$'s belong to $\mathcal{W}$
		\item[(ii)] the span is closed under scalar multiplication because
			\begin{equation*}
				\lambda(\lambda_1 v_1+\dots+\lambda_n v_n)=\left(
				\underbrace{(\lambda\lambda_1)}_{\in\mathcal{F}} v_1+\dots+
				\underbrace{(\lambda\lambda_n)}_{\in\mathcal{F}} v_n
				\right)\in\mathcal{W}
			\end{equation*}
		\item[(iii)] the span is closed under addition because
			\begin{equation*}
				\sum_{i=1}^n \lambda_i v_i+\sum_{i=1}^n \mu_i v_i=
				\left(
				\sum_{i=1}^n \underbrace{(\lambda_i+\mu_i)}_{\in\mathcal{F}}v_i
				\right)\in\mathcal{W}
			\end{equation*}
	\end{enumerate}
	Note that $v_i$'s in $\fspan\{v_1, \dots, v_n\}$ are also called a \textit{spanning set}
	of $\mathcal{W}$. Furthermore, $\mathcal{W}$ is the smallest subspace that
	contains the elements $v_1, \dots, v_n$.
\end{proof}

\begin{exm}
	\begin{flushleft}
		Another way of writing a system $Ax=b$ is the following: Denote the
		columns of $A$ by $A_1, A_2, \dots, A_n$ for $n\in\mathbb{N}$, then
		\begin{equation*}
			x_1 A_1 + \dots x_n A_n = b
		\end{equation*}
		The system $Ax=b$ then has a solution if and only if $b$ is a linear
		combination of the columns of $A$, i.e. $b\in\fspan\{v_1,\dots,v_n\}=\fcol(A)$.
		In particular this also applies to \pref{example}{exm-system-of-linear-equations}.
	\end{flushleft}
\end{exm}

\begin{definition}\label{def-matrix-cols-rows}
	Let $A$ be a matrix. Then we define
	\begin{enumerate}
		\item $\fcol(A)\defines\fspan\{\text{columns of }A\}$
		\item $\frow(A)\defines\fspan\{\text{rows of }A\}$
	\end{enumerate}
\end{definition}

\begin{exm}
	Let $u_1=\inlinematrix{-3\\1\\-1}$, and $u_2=\inlinematrix{5\\-2\\1}$. Show that
	\begin{equation*}
		v=\inlinematrix{1\\-1\\-1}\in\fspan\left\{u_1,u_2\right\}=\mathcal{W}
	\end{equation*}
	\begin{flushleft}
		\textbf{Answer}:
		This question is equivalent to solving the system of linear equations in
		\begin{equation*}
			v=x u_1 + y u_2
		\end{equation*}
		for some scalars $x,y\in\mathbb{R}$. If we plug in the vectors for $u_1,u_2$
		and rewrite this system into its augmented matrix form we can find a
		solution for this system by transforming it into the reduced row-echelon
		form:
		\begin{align*}
			\implies & \begin{pmatrix}
				-3 & 5  \\
				1  & -2 \\
				-1 & 1
			\end{pmatrix}
			\begin{pmatrix}
				x \\ y
			\end{pmatrix}=
			\begin{pmatrix}
				1  \\
				-1 \\
				-1
			\end{pmatrix}                            \\
			\Longrightarrow
			         & \begin{pmatrix}[cc|c]
				-3 & 5  & 1  \\
				1  & -2 & -1 \\
				-1 & 1  & -1
			\end{pmatrix}                 \\
			\xRightarrow{\substack{T_{12}}}
			         & \begin{pmatrix}[cc|c]
				1  & -2 & - 1 \\
				-3 & 5  & 1   \\
				-1 & 1  & -1
			\end{pmatrix}                 \\
			\xRightarrow{\substack{L_{21}(3)                      \\ L_{31}(1)}}
			         & \begin{pmatrix}[cc|c]
				1 & -2 & 1  \\
				0 & -1 & -2 \\
				0 & -1 & -2
			\end{pmatrix}                 \\
			\xRightarrow{\substack{D_2(-1)                        \\ L_{32}(1)}}
			         & \begin{pmatrix}[cc|c]
				1 & -2 & 1 \\
				0 & 1  & 2 \\
				0 & 0  & 0
			\end{pmatrix}                 \\
			\xRightarrow{\substack{L_{21}(2)}}
			         & \begin{pmatrix}[cc|c]
				1 & 0 & 3 \\
				0 & 1 & 2 \\
				0 & 0 & 0
			\end{pmatrix}\implies x=2,y=3 \\
		\end{align*}
		Turns out that this system did have a unique solutions, therefore
		\begin{equation*}
			\begin{pmatrix}
				1 \\-1\\-1
			\end{pmatrix}=3
			\begin{pmatrix}
				-3 \\1\\-1
			\end{pmatrix}+2
			\begin{pmatrix}
				5 \\-2\\1
			\end{pmatrix}
		\end{equation*}
		\textit{Note: Questions about spans and linear combinations can be
			translated to systems of linear equations and vice versa.}
	\end{flushleft}
\end{exm}

\begin{exm}\label{exm-span-of-r2}
	\begin{equation*}
		\mathbb{R}^2=\fspan\left\{
		\begin{pmatrix}
			1 \\0
		\end{pmatrix},
		\begin{pmatrix}
			0 \\1
		\end{pmatrix}
		\right\}
	\end{equation*}
\end{exm}
