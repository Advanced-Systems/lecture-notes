\subsection{Limit Prerequisites}\label{subsec-limit-prerequisites}

\begin{definition}\label{def-interval-notation}
    As opposed to closed intervals, open and half-open intervals contain infinitely
    many elements. The $\infty$ symbol always needs to be paired with a parenthesis
    rather than a bracket because it is technically incorrect to think of infinity
    as a number. So far, we haven't agreed on a formal definition for the concept
    of infinity, but it will become important later when we discuss limits in more
    detail.
    \begin{enumerate}
        \item $(a,b)\defines\left\{x\setbuild a < x < b\right\}       \quad \text{(Open interval)}$
        \item $(a,b]\defines\left\{x\setbuild a < x \leq b\right\}    \quad \text{(Half-open interval)}$
        \item $[a,b)\defines\left\{x\setbuild a \leq x < b\right\}    \quad \text{(Half-closed interval)}$
        \item $[a,b]\defines\left\{x\setbuild a \leq x \leq b\right\} \quad \text{(Closed interval)}$
    \end{enumerate}
\end{definition}

\begin{definition}\label{def-epsilon-neighborhood}
    Let $a\in\mathbb{R}$ and $\varepsilon > 0$. Then the open interval 
    $(a-\varepsilon,a+\varepsilon)$ is called the epsilon neighborhood of $a$
    and is denoted by
    \begin{equation}
        \mathcal{U}_\varepsilon(a)\defines
        \left\{x \setbuild x\in\mathbb{R},\abs{x-a}<\varepsilon \right\}=
        (a-\varepsilon,a+\varepsilon)
    \end{equation}
\end{definition}

\begin{rem}\label{rem-epsilon-neighborhood}
    In reference to property 8 of theorem (\ref{thm-absolute-value-properties})
    we can note that
    \begin{equation}
        x\in(a-\varepsilon,a+\varepsilon) \iff \abs{x-a} < \varepsilon
    \end{equation}
\end{rem}

\begin{definition}\label{def-epsilon-punctured-neighborhood}
    Let $a\in\mathbb{R}$. For $\varepsilon>0$ we define the punctured epsilon 
    neighborhood of $a$, denoted by 
    \begin{equation}
        \mathcal{U}_{\varepsilon}^{\bolddot}(a)\defines
        \left\{x \setbuild x\in\mathbb{R}, 0<\abs{x-a}<\varepsilon\right\}=
        (a-\varepsilon,a+\varepsilon)\setminus\{a\}
    \end{equation}
\end{definition}

\begin{definition}\label{def-limit-point}
    Let $M\subset\mathbb{R}$ and $a\in\mathbb{R}$. Then, $a$ is called a limit 
    point\footnote{Or cluster point} of $M$ \textit{iff}
    \begin{equation}
        \bigwedge_{\varepsilon>0}(M\cap\mathcal{U}_\varepsilon^{\bolddot}\neq\emptyset)
    \end{equation}
\end{definition}

\begin{definition}\label{def-isolated-point}
    In contrast to definition (\ref{def-limit-point}), we call $a$ an isolated point
    of $M$ \textit{iff}
    \begin{equation}
        \neg\bigwedge_{\varepsilon>0}(M\cap\mathcal{U}_\varepsilon^{\bolddot}\neq\emptyset)
        \iff
        \bigvee_{\varepsilon>0}(M\cap\mathcal{U}_\varepsilon^{\bolddot}=\emptyset)
    \end{equation}
    \textit{i.e.} $a\in M\subset\mathbb{R}$ and $a$ is not a limit point.
\end{definition}

\begin{rem}
    If $a$ is an isolated point of $M$, then \cite[p.67]{wuest2009}
    \begin{align*}
        a\in M:\mathcal{U}_\varepsilon(a)
        &=(\{a\}\cup\mathcal{U}_\varepsilon^{\bolddot})\cap M \\
        &=(\{a\}\cap M)\cup(\mathcal{U}_\varepsilon^{\bolddot}\cap M) \\
        &=\{a\}
    \end{align*}
    Therefore, in the neighborhood of $a$ are no other elements of the set $M$.
\end{rem}

\begin{exm}\label{exm-limit-points}
    Find the set of all limit points for
    \begin{enumerate}
        \item $A\defines\displaystyle\bigcup_{n\in\mathbb{N}}\left(\frac{1}{n},2-\frac{1}{n}\right)$
        \item $B\defines\left\{x\in\mathbb{R}\setbuild x=n+\frac{1}{m}\quad\text{($n,m$ appropriate)}\right\}$
    \end{enumerate}
    \begin{flushleft}
        \textbf{\nth{1} Answer:} TODO
    \end{flushleft}
    \begin{flushleft}
        \textbf{\nth{2} Answer:} TODO
    \end{flushleft}
\end{exm}

\begin{definition}\label{def-bounded-sets}
    Let $A \subseteq\mathbb{R}$ be a set.
    \begin{enumerate}
        \item Then $A$ is called bounded from above if there exists $M\in\mathbb{R}$ 
        such that $x\leq M$ for any $x\in A$.
        \item Similarly, $A$ is bounded from below if there exists $m\in\mathbb{R}$ 
        such that $x\geq M$ for any $x\in A$.
        \item Finally, $A$ is called bounded if it is bounded from above and below.
        \footnote{Notice that the boundary points in this definition
        lay no claim to uniqueness.}
    \end{enumerate}
\end{definition}

\begin{exm}\label{exm-bounded-sets:1}
    \hfill
    \begin{enumerate}
        \item Consider the set of natural numbers: $\mathbb{N}$ is not bounded from above, but 
        below, \textit{i.e.} $1$ is a lower bound of $\mathbb{N}$.
        \item Let $A=(-3,2]$. Then this set is bounded from above and below.
        \item Let $B=\left\{\frac{1}{n}\setbuild n\in\mathbb{N}\right\}$. Then this
        set is bounded from above by $1$, and bounded from below by $0$.
    \end{enumerate}
\end{exm}

\begin{definition}\label{def-supremum-infimum-sets}
    Let $A\subset\mathbb{R}$ be a set.
    \begin{enumerate}
        \item $S$ is called the supremum of $A$ if it is the smallest upper bound 
        of $A$, and is denoted by $S=\sup(A)$.
        \item $I$ is called the infimum of $A$ if it is the largest lower bound 
        of $A$, and is denoted by $I=\inf(A)$.
    \end{enumerate}
\end{definition}

\begin{exm}\label{exm-bounded-sets:2}
    Consider the sets from \pref{example}{exm-bounded-sets:1}. Then\footnote{Remark:
    $\mathbb{N}\defines\{1,2,3,\dots\}$}
    \begin{enumerate}
        \item $\inf(\mathbb{N})=1$
        \item $\inf(A)=-3$ and $\sup(A)=2$
        \item $\inf(B)=0$ and $\sup(B)=1$
    \end{enumerate}
\end{exm}

\begin{definition}\label{def-maximum-minimum}
    Let $A\subset\mathbb{R}$ be a set.
    \begin{enumerate}
        \item If $S=\sup(A)$, then $S\in A$ is also a maximum of $A$ and is denoted by $S=\max(A)$.
        \item If $I=\inf(A)$, then $I\in A$ is also a minimum of $A$ and is denoted by $I=\min(A)$.
    \end{enumerate}
\end{definition}

\begin{exm}\label{exm-bounded-sets:3}
    Expanding on the results from example (\ref{exm-bounded-sets:2}), we note that
    \begin{enumerate}
        \item $\min(\mathbb{N})=1$
        \item $\min(A)=\text{\gls{dne}}$ and $\max(A)=2$
        \item $\min(B)=\text{\gls{dne}}$ and $\max(B)=1$
    \end{enumerate}
\end{exm}
