\subsection{Limit Prerequisites}\label{subsec-limit-prerequisites}

\begin{definition}\label{def-interval-notation}
    As opposed to closed intervals, open and half-open intervals contain infinitely
    many elements. The $\infty$ symbol always needs to be paired with a parenthesis
    rather than a bracket because it is technically incorrect to think of infinity
    as a number. So far, we haven't agreed on a formal definition for the concept
    of infinity, but it will become important later when we discuss limits in more
    detail.
    \begin{enumerate}
        \item $(a,b)\defines\left\{x\setbuild a < x < b\right\}       \quad \text{(Open interval)}$
        \item $(a,b]\defines\left\{x\setbuild a < x \leq b\right\}    \quad \text{(Half-open interval)}$
        \item $[a,b)\defines\left\{x\setbuild a \leq x < b\right\}    \quad \text{(Half-closed interval)}$
        \item $[a,b]\defines\left\{x\setbuild a \leq x \leq b\right\} \quad \text{(Closed interval)}$
    \end{enumerate}
\end{definition}

\begin{definition}\label{def-epsilon-neighborhood}
    Let $a\in\mathbb{R}$ and $\varepsilon > 0$. Then the open interval 
    $(a-\varepsilon,a+\varepsilon)$ is called the epsilon neighborhood of $a$
    and is denoted by
    \begin{equation}
        \mathcal{U}_\varepsilon(a)\defines
        \left\{x \setbuild x\in\mathbb{R},\abs{x-a}<\varepsilon \right\}=
        (a-\varepsilon,a+\varepsilon)
    \end{equation}
\end{definition}

\begin{rem}\label{rem-epsilon-neighborhood}
    In reference to property 8 of theorem (\ref{thm-absolute-value-properties})
    we can note that
    \begin{equation}
        x\in(a-\varepsilon,a+\varepsilon) \iff \abs{x-a} < \varepsilon
    \end{equation}
\end{rem}

\begin{definition}\label{def-epsilon-punctured-neighborhood}
    Let $a\in\mathbb{R}$. For $\varepsilon>0$ we define the punctured epsilon 
    neighborhood of $a$, denoted by 
    \begin{equation}
        \mathcal{U}_{\varepsilon}^{\bolddot}(a)\defines
        \left\{x \setbuild x\in\mathbb{R}, 0<\abs{x-a}<\varepsilon\right\}=
        (a-\varepsilon,a+\varepsilon)\setminus\{a\}
    \end{equation}
\end{definition}

\begin{definition}\label{def-limit-point}
    Let $M\subset\mathbb{R}$ and $a\in\mathbb{R}$. Then, $a$ is called a limit 
    point\footnote{Or cluster point} of $M$ \textit{iff}
    \begin{equation}
        \bigwedge_{\varepsilon>0}(M\cap\mathcal{U}_\varepsilon^{\bolddot}\neq\emptyset)
    \end{equation}
\end{definition}

\begin{definition}\label{def-isolated-point}
    In contrast to definition (\ref{def-limit-point}), we call $a$ an isolated point
    of $M$ \textit{iff}
    \begin{equation}
        \neg\bigwedge_{\varepsilon>0}(M\cap\mathcal{U}_\varepsilon^{\bolddot}\neq\emptyset)
        \iff
        \bigvee_{\varepsilon>0}(M\cap\mathcal{U}_\varepsilon^{\bolddot}=\emptyset)
    \end{equation}
    \textit{i.e.} $a\in M\subset\mathbb{R}$ and $a$ is not a limit point.
\end{definition}

\begin{rem}
    If $a$ is an isolated point of $M$, then \cite[p.67]{wuest2009}
    \begin{align*}
        a\in M:\mathcal{U}_\varepsilon(a)
        &=(\{a\}\cup\mathcal{U}_\varepsilon^{\bolddot})\cap M \\
        &=(\{a\}\cap M)\cup(\mathcal{U}_\varepsilon^{\bolddot}\cap M) \\
        &=\{a\}
    \end{align*}
    Therefore, in the neighborhood of $a$ are no other elements of the set $M$.
\end{rem}

\begin{exm}\label{exm-limit-points}
    Find the set of all limit points for
    \begin{enumerate}
        \item $A\defines\displaystyle\bigcup_{n\in\mathbb{N}}\left(\frac{1}{n},2-\frac{1}{n}\right)$
        \item $B\defines\left\{x\in\mathbb{R}\setbuild x=n+\frac{1}{m}\quad\text{($n,m$ appropriate)}\right\}$
    \end{enumerate}
    \begin{flushleft}
        \textbf{\nth{1} Answer:} TODO
    \end{flushleft}
    \begin{flushleft}
        \textbf{\nth{2} Answer:} TODO
    \end{flushleft}
\end{exm}

\begin{definition}\label{def-bounded-sets}
    Let $A \subseteq\mathbb{R}$ be a set.
    \begin{enumerate}
        \item Then $A$ is called bounded from above if there exists $M\in\mathbb{R}$ 
        such that $x\leq M$ for any $x\in A$.
        \item Similarly, $A$ is bounded from below if there exists $m\in\mathbb{R}$ 
        such that $x\geq M$ for any $x\in A$.
        \item Finally, $A$ is called bounded if it is bounded from above and below.
        \footnote{Notice that the boundary points in this definition
        lay no claim to uniqueness.}
    \end{enumerate}
\end{definition}

\begin{exm}\label{exm-bounded-sets:1}
    \hfill
    \begin{enumerate}
        \item Consider the set of natural numbers: $\mathbb{N}$ is not bounded from above, but 
        below, \textit{i.e.} $1$ is a lower bound of $\mathbb{N}$.
        \item Let $A=(-3,2]$. Then this set is bounded from above and below.
        \item Let $B=\left\{\frac{1}{n}\setbuild n\in\mathbb{N}\right\}$. Then this
        set is bounded from above by $1$, and bounded from below by $0$.
    \end{enumerate}
\end{exm}

\begin{definition}\label{def-supremum-infimum-sets}
    Let $A\subset\mathbb{R}$ be a set.
    \begin{enumerate}
        \item $S$ is called the supremum of $A$ if it is the smallest upper bound 
        of $A$, and is denoted by $S=\sup(A)$.
        \item $I$ is called the infimum of $A$ if it is the largest lower bound 
        of $A$, and is denoted by $I=\inf(A)$.
    \end{enumerate}
\end{definition}

\begin{exm}\label{exm-bounded-sets:2}
    Consider the sets from \pref{example}{exm-bounded-sets:1}. Then\footnote{Remark:
    $\mathbb{N}\defines\{1,2,3,\dots\}$}
    \begin{enumerate}
        \item $\inf(\mathbb{N})=1$
        \item $\inf(A)=-3$ and $\sup(A)=2$
        \item $\inf(B)=0$ and $\sup(B)=1$
    \end{enumerate}
\end{exm}

\begin{definition}\label{def-maximum-minimum}
    Let $A\subset\mathbb{R}$ be a set.
    \begin{enumerate}
        \item If $S=\sup(A)$, then $S\in A$ is also a maximum of $A$ and is denoted by $S=\max(A)$.
        \item If $I=\inf(A)$, then $I\in A$ is also a minimum of $A$ and is denoted by $I=\min(A)$.
    \end{enumerate}
\end{definition}

\begin{exm}\label{exm-bounded-sets:3}
    Expanding on the results from example (\ref{exm-bounded-sets:2}), we note that
    \begin{enumerate}
        \item $\min(\mathbb{N})=1$
        \item $\min(A)=\text{\gls{dne}}$ and $\max(A)=2$
        \item $\min(B)=\text{\gls{dne}}$ and $\max(B)=1$
    \end{enumerate}
\end{exm}

\begin{definition}\label{def-monotonicity}
    Let $x,y\in\mathbb{R}$. A function is called
    \begin{enumerate}
        \item monotonically increasing, if for all $x,y$ it follows that
        \begin{equation}\label{eq-monotonically-increasing}
            x<y \implies f(x)\leq f(y)
        \end{equation}
        \item strictly monotonically increasing, if for all $x,y$ follows that
        \begin{equation}\label{eq-strictly-monotonically-increasing}
            x<y \implies f(x)<f(y)
        \end{equation}
        \item monotonically decreasing, if for all $x,y$ follows that
        \begin{equation}\label{eq-monotonically-decreasing}
            x<y \implies f(x)\geq f(y)
        \end{equation}
        \item strictly monotonically decreasing, if for all $x,y$ follows that
        \begin{equation}\label{eq-strictly-monotonically-decreasing}
            x<y \implies f(x)>f(y)
        \end{equation}
    \end{enumerate}
\end{definition}

\begin{exm}
    Let $f(x)=\sin(x)$.
    \begin{enumerate}
        \item Then $f$ is strictly increasing on $[-\tfrac{\pi}{2},\tfrac{\pi}{2}]$.
        \item Then $f$ is strictly decreasing on $[\tfrac{\pi}{2},\tfrac{3\pi}{2}]$.
    \end{enumerate}
\end{exm}

\begin{definition}\label{def-even-function}
    Let $f$ be a function. Then, $f$ is called an even function if
    \begin{equation}
        \forall x\in\domain{f}:f(x)=f(-x)
    \end{equation}
\end{definition}

\begin{exm}
    See the list below for some even functions:
    \begin{itemize}
        \item $x\mapsto\abs{x}$
        \item $x\mapsto x^2$
        \item $x\mapsto\cos(x)$
    \end{itemize}
\end{exm}

\begin{definition}\label{def-odd-function}
    Let $f$ be a function. Then, $f$ is called an odd function if
    \begin{equation}
        \forall x\in\domain{f}:f(x)=-f(-x)
    \end{equation}
\end{definition}

\begin{exm}
    See the list below for some odd functions:
    \begin{itemize}
        \item $x\mapsto x$
        \item $x\mapsto x^3$
        \item $x\mapsto\sin(x)$
    \end{itemize}
\end{exm}

\begin{definition}
    Let $f$ be a function. Then, $f$ is called a periodic function if
    \begin{equation}
        \forall x\in\domain{f}\,\exists T\in\mathbb{R}: f(x)=f(x+T)
    \end{equation}
\end{definition}

\begin{exm}
    See the list below for some periodic functions:
    \begin{itemize}
        \item $x\mapsto\sin(x)$
        \item $x\mapsto\cos(x)$
        \item $x\mapsto\tan(x)$
        \item \hyperref[def-dirichlet-function]{$x\mapsto D(x)$}
    \end{itemize}
\end{exm}

\begin{definition}\label{def-bounded-function}
    Let $f$ be a function. Then, $f$ is called\footnote{This definition extends 
    the notion of boundaries as defined in \pref{definition}{def-bounded-sets} to 
    functions.}
    \begin{enumerate}
        \item bounded from above if
        \begin{equation}
            \forall x\in\domain{f}\,\exists M\in\mathbb{R}:f(x)\leq M
        \end{equation}
        \item bounded from below if 
        \begin{equation}
            \forall x\in\domain{f}\,\exists m\in\mathbb{R}:f(x)\geq m
        \end{equation}
        \item bounded if $f$ is bounded from above and below.
    \end{enumerate}
\end{definition}

\begin{exm}
    The function $f:\mathcal{D}\to\mathbb{R}$ with
    \begin{itemize}
        \item $\domain{f}=\mathbb{R}^-:f(x)=x$ is bounded from above but not below
        \item $\domain{f}=\mathbb{R}:f(x)=x^2$ is bounded from below, but not above
        \item $\domain{f}=\mathbb{R}^+:f(x)=\sqrt{x}$ is bounded from below, but not above
        \item $\domain{f}=\mathbb{R}:f(x)=\sin(x)$ is bounded from above and below
        \item $\domain{f}=\mathbb{R}:f(x)=\ffloor(x)$ is neither bounded from above nor below
    \end{itemize}
\end{exm}

\begin{rem}\label{rem-bounded-function}
    A function $f$ is bounded \textit{iff}
    \begin{equation}
        \forall x\in\domain{f}\,\exists B\in\mathbb{R}:\abs{f(x)}\leq B
    \end{equation}
    since $\abs{f(x)}\leq B\iff -B\leq f(x) \leq B$ by the \nth{8} property from
    \pref{theorem}{thm-absolute-value-properties}.
\end{rem}

\begin{definition}\label{def-supremum-infimum-functions}
    Let $f$ be a function. Then we define
    \begin{enumerate}
        \item the supremum of a function by
        \begin{equation}
            \sup_{x\in\mathcal{D}}(f)\defines\sup\left\{f(x)\setbuild x\in\mathcal{D}\right\}
        \end{equation}
        \item the infimum of a function by
        \begin{equation}
            \inf_{x\in\mathcal{D}}(f)\defines\inf\left\{f(x)\setbuild x\in\mathcal{D}\right\}
        \end{equation}
    \end{enumerate}
\end{definition}

\begin{exm}
    Let $f(x)=\arctan(x)$. Then,
    \begin{itemize}
        \item the supremum of this function is
        \begin{equation}
            \sup(f)=\frac{\pi}{2}
        \end{equation}
        \item the infimum of this function is
        \begin{equation}
            \inf(f)=-\frac{\pi}{2}
        \end{equation}
    \end{itemize}
\end{exm}

\begin{definition}\label{def-function-operations}
    Let $f$ and $g$ be two functions. Then we define
    \begin{enumerate}
        \item the addition between two functions by
        \begin{equation}
            (f \pm g)(x) \defines f(x) \pm g(x)
        \end{equation}
        \item the multiplication between two functions by
        \begin{equation}
            (f \cdot g)(x) \defines f(x) \cdot g(x)
        \end{equation}
        \item the division between two functions by\footnote{For $g(x)\neq0$}
        \begin{equation}
            \left(\frac{f}{g}\right)(x) \defines \frac{f(x)}{g(x)} 
        \end{equation}
        \item the composition between two functions by
        \begin{equation}
            (f \circ g)(x) \defines f(g(x))
        \end{equation}
    \end{enumerate}
\end{definition}

\begin{exm}
    Let $f:\mathbb{R}\to[-1,1],x\mapsto\sin(x)$ and $g:\mathbb{R}\setminus\{0\}\to\mathbb{R},x\mapsto\tfrac{1}{x}$.
    Then the two compositions for these functions are given by
    \begin{align*}
        &(f \circ g):\mathbb{R}\setminus\{0\}\to[-1,1],x\mapsto\sin\left(\frac{1}{x}\right)\\
        &(g \circ f):\mathbb{R}\setminus\left\{\pi k\setbuild k\in\mathbb{Z}\right\}\to\mathbb{R},x\mapsto\frac{1}{\sin(x)}
    \end{align*}
\end{exm}

\begin{rem}\label{rem-monotone-implies-injective}
    If $f$ is strictly monotone, then $f$ is injective.
\end{rem}

\begin{rem}\label{rem-elementary-functions}
    It is advised to remember the following elementary functions by heart:
    \begin{itemize}
        \item Polynomial functions\footnote{$a_0$ is the so-called \enquote{free coefficient}}: $p(x)=\sum_{i=0}^na_ix^i$
        \item Rational functions: $f(x)=\tfrac{p(x)}{q(x)}$ where $p(x)$ and $q(x)$ are well-defined polynomial functions
        \item Exponential functions: $f(x)=a^x$ for $a>0$ and $a\neq 1$
        \item Trigonometric functions: for example, $f(x)=\sin(x)$
        \item Inverse trigonometric functions: for example, $f^{-1}(x)=\arcsin(x)$
        \item Inverse functions of polynomials, for example $f^{-1}(x)=\sqrt{x}$
        \item Inverse functions of rationals
        \item Inverse functions of exponentials: for example, $f^{-1}(x)=\log_a(x)$
    \end{itemize}
\end{rem}

\begin{definition}\label{def-elementary-functions}
    An elementary function is any function obtained from the list in \pref{remark}{rem-elementary-functions},
    including their combinations using the operations defined in \pref{definition}{def-function-operations}.
    \footnote{Note that this is not a standard definition.}
\end{definition}
